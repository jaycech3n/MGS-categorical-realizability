\documentclass[a4paper, oneside, 12pt, openany]{book}
\pdfoutput=1

\usepackage{packages}
\usepackage{packages-main}
\addbibresource{references.bib}
\usepackage{macros}

% Prevent footnotes over multiple pages
\interfootnotelinepenalty=10000

% Label tables just like equations, theorems, definitions, etc.
%
% NB: This can be confusing if LaTeX does not place the table at the point of
% writing (e.g. for lack of space)!
\numberwithin{equation}{chapter}
\numberwithin{table}{chapter}
\makeatletter
\let\c@equation\c@table
\makeatother

% Setting up the coloured environments
%
\newbool{shade-envs}
% This can be used to toggle the coloured environments on or off.
\setboolean{shade-envs}{true}

%%
\ifthenelse{\boolean{shade-envs}}{%
  % Colours are as in Andrej Bauer's notes on realizability:
  % https://github.com/andrejbauer/notes-on-realizability
  \colorlet{ShadeOfPurple}{blue!5!white}
  \colorlet{ShadeOfYellow}{yellow!5!white}
  \colorlet{ShadeOfGreen} {green!5!white}
  \colorlet{ShadeOfBrown} {brown!10!white}
  % But we also shade proofs
  \colorlet{ShadeOfGray}  {gray!10!white}
  % For exercises
  \colorlet{ShadeOfRed}   {red!10!white}
}
% If we don't want to have shaded environments, then we use a closing symbol
% \lozenge to mark the end of remarks, definitions and examples.
{%
  \declaretheoremstyle[
      spaceabove=6pt,
      spacebelow=6pt,
      bodyfont=\normalfont,
      qed=\(\lozenge\)
  ]{definitionwithbox}
  \declaretheoremstyle[
      headfont=\itshape,
      bodyfont=\normalfont,
      qed=\(\lozenge\)
      ]{remarkwithbox}
}

% Now we set the shading using the tcolorbox package.
%
% The related thmtools' option "shaded" and the package mdframed seem to have
% issues: the former does not allow for page breaks in shaded environments and
% the latter puts double spacing between two shaded environments.
\tcbset{shadedenv/.style={
    colback={#1},
    frame hidden,
    enhanced,
    breakable,
    boxsep=0pt,
    left=2mm,
    right=2mm,
    % LaTeX thinks this is too wide (as becomes clear from the many "Overfull
    % \hbox" warnings, but optically it looks spot on.
    add to width=1.1mm,
    enlarge left by=-0.6mm}
}

% Keep a count of the number of exercises
\newtotcounter{allexercises}

\ifthenelse{\boolean{shade-envs}}{%
  \declaretheorem[sibling=equation]{theorem}
  \declaretheorem[unnumbered,title=Theorem]{theorem*}
  \declaretheorem[sibling=theorem]{lemma,proposition,corollary}
  \declaretheorem[unnumbered,title=Lemma]{lemma*}
  \declaretheorem[sibling=theorem,style=definition]{definition}
  \declaretheorem[sibling=theorem,style=definition]{example}
  \declaretheorem[sibling=theorem,style=definition]{notation}
  \declaretheorem[sibling=theorem,style=remark]{remark}
  \declaretheorem[sibling=theorem,style=definition,refname={Exercise,Exercises},postheadhook=\stepcounter{allexercises}]{exercise}
  %
  \tcolorboxenvironment{theorem}    {shadedenv={ShadeOfPurple}}
  \tcolorboxenvironment{theorem*}   {shadedenv={ShadeOfPurple}}
  \tcolorboxenvironment{lemma}      {shadedenv={ShadeOfPurple}}
  \tcolorboxenvironment{lemma*}     {shadedenv={ShadeOfPurple}}
  \tcolorboxenvironment{proposition}{shadedenv={ShadeOfPurple}}
  \tcolorboxenvironment{corollary}  {shadedenv={ShadeOfPurple}}
  \tcolorboxenvironment{definition} {shadedenv={ShadeOfYellow}}
  \tcolorboxenvironment{notation} {shadedenv={ShadeOfYellow}}
  \tcolorboxenvironment{example}    {shadedenv={ShadeOfGreen}}
  \tcolorboxenvironment{remark}     {shadedenv={ShadeOfBrown}}
  \tcolorboxenvironment{proof}      {shadedenv={ShadeOfGray}}
  \tcolorboxenvironment{exercise}   {shadedenv={ShadeOfRed}}
}{% Use closing symbols if we don't have colours
  \declaretheorem[sibling=equation]{theorem}
  \declaretheorem[sibling=theorem]{lemma,proposition,corollary}
  \declaretheorem[unnumbered,title=Theorem]{theorem*}
  \declaretheorem[unnumbered,title=Lemma]{lemma*}
  \declaretheorem[sibling=theorem,style=definitionwithbox]{definition}
  \declaretheorem[sibling=theorem,style=definitionwithbox]{notation}
  \declaretheorem[sibling=theorem,style=definitionwithbox]{example}
  \declaretheorem[sibling=theorem,style=remarkwithbox]{remark}
  \declaretheorem[sibling=theorem,style=definitionwithbox,postheadhook=\stepcounter{allexercises}]{exercise}
  \tcolorboxenvironment{theorem}    {shadedenv={white}}
  \tcolorboxenvironment{theorem*}   {shadedenv={white}}
  \tcolorboxenvironment{lemma}      {shadedenv={white}}
  \tcolorboxenvironment{lemma*}     {shadedenv={white}}
  \tcolorboxenvironment{proposition}{shadedenv={white}}
  \tcolorboxenvironment{corollary}  {shadedenv={white}}
  \tcolorboxenvironment{definition} {shadedenv={white}}
  \tcolorboxenvironment{notation} {shadedenv={white}}
  \tcolorboxenvironment{example}    {shadedenv={white}}
  \tcolorboxenvironment{remark}     {shadedenv={white}}
  \tcolorboxenvironment{proof}      {shadedenv={white}}
  \tcolorboxenvironment{exercise}   {shadedenv={white}}
  }
  \declaretheorem[sibling=theorem,style=remark,numbered=no]{claim}

% Note that proofs will still have the \qed symbol at the end, even when shaded,
% because we prefer to keep up the tradition.


\begin{document}

\frontmatter

\begin{titlepage}
\begin{center}
  \vspace*{\stretch{0.5}}

  \large % Default size for the title page

  {\Huge\textsc{Categorical Realizability}\par}

  \vspace{\stretch{0.2}}

  by

  \vspace{\stretch{0.2}}

  {\huge\textsc{Tom de Jong}}

  \vspace{\stretch{0.5}}

  {\Large{Lecture notes and exercises for the\\
      \textsc{Midlands Graduate School (MGS)}}} \\
  \vspace{\stretch{0.1}}
  8--12 April 2024, Leicester, UK

  \vspace{\stretch{1}}

  % \begingroup
  % \tikzset{every picture/.style={color=Gray!90!Black}}
  % \begin{tikzcd}
  %   0 & 1 & 2 & 3 & \ldots \\
  %   & & \bot \ar[ull,no head] \ar[ul,no head] \ar[u,no head] \ar[ur, no head] \ar[urr, no head]
  % \end{tikzcd}
  % \endgroup

  \makeicon

  \vfill

  \flushright
  {\normalsize{School of Computer Science \\
  University of Nottingham \\
  January--March 2024}}

\end{center}
\end{titlepage}

%%% Local Variables:
%%% mode: latexmk
%%% TeX-master: "../main"
%%% End:
\include{frontmatter/dedication}
\restoregeometry%

\include{frontmatter/abstract}
\chapter{Acknowledgements}

It is my pleasure to express my sincere thanks to Jaap van Oosten---to whom I
have dedicated these notes---for teaching an excellent course on category
theory~\cite{vanOosten2016} and introducing me to realizability when I was a
master's student in mathematics at Utrecht University back in 2015--2018.

I thank Ignacio Bellas, Rahul Chhabra, Josh Chen, Stefania Damato, Johnson He,
Chris Purdy, Vincent Rahli, Alyssa Renata, Jingjie Yang and Mark Williams for
pointing out or fixing typos, and Mart\'in Escard\'o for a discussion on the BHK
interpretation.
%
I~am also grateful to Josh Chen for volunteering to be a teaching assistant for
the course.
%
For the illustration on the title page I adapted \verb|tikz| code from
\emph{AboAmmar}'s answer on \TeX\ StackExchange~\cite{latex-triangle}.

%%% Local Variables:
%%% mode: latexmk
%%% TeX-master: "../main"
%%% End:


\setcounter{tocdepth}{2}
\tableofcontents

\mainmatter%

\chapter{Introduction}

\emph{Realizability} originates with Kleene~\cite{Kleene1945} in the context of
proof theory and sought to make a precise connection between intuitionistic
(constructive) mathematics and computability theory.
%
This led to an effective interpretation of intuitionistic number theory using
computability theory similar to the Brouwer--Heyting--Kolmogorov (BHK)
interpretation\footnote{See \cite[\S3.1 and \S5.3, Sec.~3 and 5,
  Ch.~1]{TroelstraVanDalen1988} for references and a brief discussion of the
  BHK interpretation, and \cite[p.~241]{vanOosten2002} for a historical
  discussion of its relevance to Kleene's realizability.} from the 1920--30's but
with the fundamental difference that %, unlike the informal BHK interpretation,
Kleene's interpretation was a formal mathematical interpretation.

In the 1980's, Hyland introduced the \emph{effective topos}: a category whose
logic is governed by Kleene's realizability.
%
Intuitively, the effective topos presents us with an alternative world of
mathematics where---unlike in the category of sets and functions---``everything is
computable''.
%
Actually, the effective topos is an instance of a general class of categories
known as \emph{realizability toposes} as originally developed by Hyland,
Johnstone and Pitts~\cite{HJP1980,Pitts1981}.

Every realizability topos is parametrized by an abstract model of computation,
known as a \emph{partial combinatory algebra (pca)}. Classical computability
theory based on partial Turing computable functions on natural numbers gives a
pca known as \emph{Kleene's first model} and the resulting realizability topos
is Hyland's effective topos.
%
While Hyland's topos is perhaps the best understood example, many other choices
of pcas are possible with interesting realizability toposes as a
result~\cite{vanOosten2008}.

A beautiful aspect of Hyland's discovery is that it enables us to study the
interplay between category theory, logic and computability theory.
%
A typical application of realizability categories is to provide semantics to
(polymorphic) type theories like System~F (which has no direct set-theoretic
semantics~\cite{Reynolds1984}); see also \cref{sec:further-reading}.
%
Realizability has also found practical application in the form of extracting
programs from mathematical proofs, see e.g.~\cite{Minlog}.

\section{Aims}
We hope that these notes provide a self-contained and accessible introduction to
the categorical aspects of realizability for graduate students in theoretical
computer science.
%
More generally, we hope the reader will appreciate these notes as a testament of
the deep connections between category theory, logic and computability theory.

The notion of a realizability topos is fairly involved and for this reason we
focus on the simpler categories of \emph{assemblies} instead (although we
include a brief epilogue on realizability toposes).
%
While the category of assemblies lacks some features of a topos it provides more
than enough structure for our present purposes. Besides, one should arguably
first have a good grasp on the assemblies before studying realizability toposes.

\paragraph{Outline of these notes}
\begin{itemize}
\item \cref{chap:PCA} introduces partial combinatory algebra (pcas) as
  abstract models of computation and illustrate them with various examples.
  %
  Some familiarity with basic computability theory and topology is
  helpful, but certainly not required.
\item \cref{chap:assemblies} describes the category of assemblies and assembly maps
  over a fixed but arbitrary pca.
  %
  Intuitively, an assembly is a set together with computability data and an
  assembly map is a function of sets that is computable. Here, the notion of
  computability is prescribed by the pca.

  We explore the categorical structure of this category and some familiarity
  with basic category theory---say (co)limits and adjunctions---is necessary in
  most places, although I am hopeful that those unfamiliar with category theory
  can still benefit from this chapter and are perhaps inspired to learn some
  category theory.
\item
  Finally, in \cref{chap:logic}, we turn to the logical aspects of the category
  of assemblies.
  %
  We spell out the realizability interpretation of first order logic that the
  assemblies give rise to.
  %
  We illustrate the connections between category theory, logic and computability
  theory by studying certain \emph{realizability predicates} from these three
  perspectives. For example, the semidecidable predicates can be characterized
  (1)~categorically, as pullbacks of a certain two-element assembly;
  (2)~computably, as computably enumerable subsets; and (3)~logically, as those
  predicates which are presented by a binary sequence.
  %
  Finally, we exploit these connections by describing a simple result from
  \emph{synthetic computability theory}.
\end{itemize}


\section{Exercises}

The exercises are interspersed in the text, but each chapter ends with a list of
its exercises for reference.
%
There are \total{allexercises} exercises in total.

\section{References}

In preparing these notes I have mainly used the standard
textbook~\cite{vanOosten2008} by van Oosten, as well as Bauer's excellent
lecture notes~\cite{Bauer2023}.
%
In a few places, e.g.\ \cref{base-change-adjoints}, I have also consulted
Streicher's notes~\cite{Streicher2018}.
%
The presentation of logic in the category of assemblies using realizability
predicates owes a lot to Bauer's treatment, although I turned to
\cite[Section~3.2.7]{vanOosten2008} for
\cref{exer:Sigma-in-Kleene-1,exer:Sigma-to-N-is-CE,exer:Sigma-ce-subsets,exer:Rice-consequence},
and included some additions in the form of
\cref{sec:revisiting-epis-monos,sec:synthetic}.
%
In notation and terminology I have stayed close to Bauer's notes as I admire its
readability.

\section{Further reading}\label{sec:further-reading}

Natural candidates for further reading are the aforementioned
notes by Bauer~\cite{Bauer2023} and Streicher~\cite{Streicher2018}, as well as
the standard textbook by van Oosten~\cite{vanOosten2008}.

The categorical semantics of (polymorphic) type theories using realizability is
treated in a variety of works, such as Reus's tutorial paper~\cite{Reus1999},
Streicher's notes~\cite{Streicher2018}, Amadio and Curien's
textbook~\cite{AmadioCurien1998} and Jacobs's textbook~\cite{Jacobs1999}; see
also the references listed on~\cite[p.~193]{vanOosten2008}.

Those looking to learn more about general models of (higher-order) computability
can consult Longley and Normann's comprehensive
textbook~\cite{LongleyNormann2015}.

To those interested in the history of realizability we recommend Troelstra's
proof-theoretic survey~\cite{Troelstra1998} and van Oosten's
essay~\cite{vanOosten2002} for its categorical aspects.

If you are interested in the formalization of mathematics, then you should look
at Chhabra's ongoing \emph{Cubical Agda} development~\cite{Chhabra2023}\footnote{With
  the caveat that the combinatory algebras are assumed to be total---at least for
  the moment.}.


%%% Local Variables:
%%% mode: latexmk
%%% TeX-master: "../main"
%%% End:

\chapter[Models of computation: partial combinatory algebras]{Models of computation: \\ partial combinatory algebras}\label{chap:PCA}

The starting ingredient in categorical realizability is a general model of
computation known as a \emph{partial combinatory algebra}, or \emph{pca} for
short.
%
The desire for a general model of computation is motivated by the many possible
interesting examples.
%
Moreover, in the next chapter, we construct the \emph{category of assemblies}
over an arbitrary pca. The construction itself works generally and is
insensitive to any particular choice of pca. However, the choice of pca is
reflected in the logical principles that the resulting category validates.
%
Thinking of categories of assemblies as worlds of computable mathematics, we can
thus build different worlds by varying our notion of pca.

Having said all this, in these notes we have chosen to work with the
\emph{untyped} notion of a pca. The \emph{typed} notion, as developed by
Longley\footnote{Fun fact: John Longley is also the author of the recent
  \emph{Castles in the Air}---an introduction to the world of mathematical logic
  cast in the form of a fantasy novel---as well as a semi-professional
  pianist.} in his PhD thesis~\cite{Longley1995}, is more general and moreover
has the advantage that it can capture (idealized) typed functional programming
languages (such as PCF~\cite{Plotkin1977,deJong2023}).
%
We only treat untyped pcas for simplicity and refer the interested reader to
Bauer's more comprehensive notes~\cite{Bauer2023}.


\begin{definition}[Partial combinatory algebra (pca), \(\kcomb\), \(\scomb\)]\label{def:pca}
  A \textbf{partial combinatory algebra} (\textbf{pca}) is a set \(\AA\)
  together with a \emph{partial} operation \(\AA \times \AA \pto \AA\), denoted
  by juxtaposition, \((\pca{a},\pca{b}) \mapsto \pca{a}\pca{b}\), such that
  there exist elements \(\kcomb\) and \(\scomb\) satisfying:
  \begin{enumerate}[(i)]
  \item\label{k-behaviour} \((\kcomb \pca{a}) \pca{b} = \pca{a}\) for all elements
    \(\pca{a},\pca{b} \in \AA\),
  \item\label{s-defined} \(((\scomb \pca{f}) \pca{g})\) is defined for all elements
    \(\pca{f},\pca{g} \in \AA\), and
  \item\label{s-behaviour}
    \(((\scomb \pca{f}) \pca{g}) \pca{a} \simeq
    (\pca{f}\pca{a})(\pca{g}\pca{a})\) for all elements
    \(\pca{f},\pca{g},\pca{a} \in \AA\).
  \end{enumerate}
  The symbol \(\simeq\) is \emph{Kleene equality} and means: either both sides
  are undefined, or both are defined and are equal elements of \(\AA\).
  %
  In particular, in~\ref{k-behaviour}, the operation \(\kcomb\pca{a}\) must be
  defined for all elements \(\pca{a} \in \AA\).
\end{definition}

We think of the elements of a pca as codes for programs that can also act as
input. Accordingly, we pronounce \(\pca{a}\pca{b}\) as ``\(\pca{a}\) applied to
\(\pca{b}\)'' and think of this as: apply the program with code \(\pca{a}\) to
input \(\pca{b}\).

In this light, the element \(\kcomb\), which we call the \emph{k-combinator}, acts
as a parameterized constant program: it takes an input \(\pca{a}\) and then
always outputs \(\pca{a}\) on any input \(\pca{b}\).

The element \(\scomb\), which we call the \emph{s-combinator}\footnote{The
  letters \emph{k} and \emph{s} come from Moses Sch\"onfinkel's combinatory
  logic~\cite{Schonfinkel1924}. They respectively come from the German words
  \emph{Konstanzfunktion} (constant function) and \emph{Verschmelzungfunktion}
  (merge function). Of course, \emph{Verschmelzungsfunktion} starts with a
  \emph{v}, but Sch\"onfinkel had to avoid confusion as there was also a
  swap-arguments combinator called the \emph{Vertauschungsfunktion}.}, acts as
parameterized application: it takes two codes for programs \(\pca{f}\) and
\(\pca{g}\) and an input \(\pca{a}\) and then applies \(\pca{f}\pca{a}\) to
\(\pca{g}\pca{a}\).

\begin{notation}
  We will economize on parentheses and write \(\pca{a}\pca{b}\pca{c}\) for
  \((\pca{a}\pca{b})\pca{c}\).
  %
  We will always use the \texttt{teletype font} for arbitrary elements of a pca
  to reinforce the idea that these should be thought of as (codes of) programs.
  %
  For fixed programming constructs, we will use this \(\pcacomb{bold\; font}\)
  instead.
\end{notation}

The point of the k- and s-combinators is that they provide us with a (very
minimal) programming interface which we will explore
in~\cref{sec:basic-programming-in-pcas}.
%
We prefer to give two examples first to strengthen our intuition for pcas.

\section{Basic examples of pcas}\label{sec:basic-examples-of-pcas}

To help build some intuition for pcas, we now consider three basic examples. The
first example is a triviality, but we include it here because the category of
assemblies (see~\cref{chap:assemblies}) over it is equivalent to the familiar category of
sets.

\begin{example}[The trivial pca]
  The \textbf{trivial pca} is the singleton set \(\set{\singleton}\) with
  application map \((\singleton,\singleton) \mapsto \singleton\).
  %
  Of course, \(\kcomb \coloneq \singleton\) and \(\scomb \coloneq \singleton\).
\end{example}

\begin{example}[Untyped \(\lambda\)-calculus as a pca, \(\Lambda\)]
  Write \(\Lambda\) for the closed terms of the untyped \(\lambda\)-calculus
  quotiented by the equivalence relation generated by
  \(\beta\)-reduction. (E.g., for a closed \(\lambda\)-term \(t\), we identify
  \((\lambda{x}.{x})t\) and \(t\).)
  %
  With \(\lambda\)-calculus application the set \(\Lambda\) forms a pca with
  \(\kcomb\) and \(\scomb\) given by the equivalence classes of
  \(\lambda{xy}.x\) and \(\lambda{xyz}.(xz)(yz)\), respectively.
\end{example}

The application function of \(\Lambda\) is actually total, i.e.\ it is defined
on any two inputs. An example of a pca with a genuine partial application---which
is also the prime example of a pca---is \emph{Kleene's first model}~\cite{Kleene1945}:

\begin{example}[Kleene's first model, \(\Kone\)]\label{ex:Kleene-1}
  We define a partial application function on the set of natural numbers
  \(\Nat\): for natural numbers \(n\) and \(m\) we take \(n\,m\) to be
  \(\prenum{n}(m)\), where \(\prenum{-}\) denotes a Turing computable
  enumeration of the Turing computable partial functions on the natural numbers.

  The existence of the k- and s-combinators follows from Kleene's
  \(S^m_n\)-theorem in computability theory, see e.g.~\cite[Section~2.5.1 and
  Theorem~2.1.5]{Bauer2023} for details.

  We write \(\Kone\) for this partial combinatory algebra and return to
  it in~\cref{chap:logic}.
\end{example}

\section{Basic programming in pcas}\label{sec:basic-programming-in-pcas}

%As previously mentioned,
The k- and s-combinators of a pca provide an
interface for writing basic programs. For example, if we put\footnote{Note
  that this indeed defines an element of \(\AA\)
  by~\cref{def:pca}\ref{s-defined}.}
\(\icomb \coloneq \scomb\kcomb\kcomb\) then \(\icomb\) acts as the
identity combinator:
\(
  \icomb\pca{a} = \scomb\kcomb\kcomb\pca{a} = \kcomb\pca{a}(\kcomb\pca{a}) =
  {\pca{a}}
  \)
  for any element \(\pca{a}\) of our pca.

While theoretically possible, it is very inconvenient to program everything
directly in terms of \(\kcomb\) and \(\scomb\). Therefore, we will
define something resembling \(\lambda\)-abstraction for our pca, allowing us to
write the clearer \(\icomb \coloneq \lambdapca{x}{\var{x}}\) instead.

\begin{definition}[Terms]% over a pca]
  We fix a countably infinite set of variables, typically denoted by
  \(x,y,z,u,v,w,x_0,x_1,\dots\), and inductively define the set of
  \textbf{terms} over a pca~\(\AA\):
  \begin{enumerate}[(i)]
  \item a variable is a term,
  \item an element of \(\AA\) is a term,
  \item given two terms \(s\) and \(t\), we may form a new term denoted by \(s\,t\).
  \end{enumerate}
  A term without variables is called \textbf{closed}.
\end{definition}

\begin{notation}
  If \(t\) is a term, then we write \(t[\pca{a_1}/x_1,\dots,\pca{a_n}/x_n]\) for
  the \emph{closed} term obtained by substituting the element
  \(\pca{a_i} \in \AA\) for the variable \(x_i\) (which may or may not occur in
  \(t\)).
\end{notation}

\begin{definition}[Defined terms]
  A closed term \(r\) is \textbf{defined} if, when we interpret all subterms of
  \(r\) of the form \({s\,t}\) as \(s\) applied to \(t\) in \(\AA\), all these
  applications are defined.

  We extend this to terms with variables: such a term \(t\) is \textbf{defined}
  if for all possible substitutions of all variables in \(t\) by elements of
  \(\AA\), the resulting closed term obtained via substitution is defined.
\end{definition}

For example, the terms \(\kcomb\scomb\icomb\) and \(\scomb x\,y\) are defined.

\begin{notation}
  We extend Kleene equality from closed terms to all terms over \(\AA\): given
  two terms \(s\) and \(t\) whose variables are among \(x_1,\dots,x_n\), we
  write \(s \simeq t\) if for all \(\pca{a_1},\dots,\pca{a_n} \in \AA\), we
  have
  \[
    s[\pca{a_1}/x_1,\dots,\pca{a_n}/x_n] \simeq
    t[\pca{a_1}/x_1,\dots,\pca{a_n}/x_n]
  \]
  as closed terms.
\end{notation}

We now define something resembling \(\lambda\)-abstraction for pcas.

\begin{definition}[``\(\lambda\)-abstraction'' in a pca, \(\lambdapca{x}{t}\)]
  For a variable \(x\) and a term \(t\), we define a new term, denoted by
  \(\lambdapca{x}{t}\), by recursion on terms:
  \begin{itemize}
  \item \(\lambdapca{x}{x} \coloneqq \icomb = \scomb\kcomb\kcomb\),
  \item \(\lambdapca{x}{y} \coloneqq \kcomb y\) if \(y\) is a variable different from \(x\),
  \item \(\lambdapca{x}{\pca{a}} \coloneqq \kcomb \pca{a}\) for \(\pca{a} \in \AA\),
  \item \(\lambdapca{x}{(t_1\,t_2)} \coloneqq \scomb\,(\lambdapca{x}{t_1})\,(\lambdapca{x}{t_2})\).
  \end{itemize}
\end{definition}

\begin{exercise}[Combinatory completeness]\label{exer:properties-of-lambda-pca}
  Prove that for every variable \(x\) and term~\(t\), the following properties
  of \(\lambdapca{x}{t}\) hold:
  \begin{enumerate}[(i)]
  \item The variables of \(\lambdapca{x}{t}\) are exactly those of \(t\) minus
    \(x\).
  \item\label{lambda-pca-def} The term \(\lambdapca{x}{t}\) is defined.
  \item For all \(\pca{a} \in \AA\), we have
    \((\lambdapca{x}{t})\pca{a} \simeq t[\pca{a}/x]\).
  \end{enumerate}
\end{exercise}

\begin{notation}
  We write \(\lambdapca{xy}{t}\) for the term
  \(\lambdapca{x}({\lambdapca{y}{t}})\) and similarly for more variables.
\end{notation}

Suppose we have a term \(t\) featuring only the variables \(x\), \(y\) and
\(z\), e.g., \(t \coloneqq z\kcomb x(\kcomb y)\).
%
We think of \(t\) as a \emph{partial} function
\(\AA \times \AA \times \AA \pto \AA\), taking three inputs \(\pca{a}\),
\(\pca{b}\) and \(\pca{c}\) that we may substitute in \(t\) for \(x\), \(y\) and
\(z\), respectively, and that results in the term
\(\pca{c}\kcomb\pca{a}(\kcomb\pca{b})\) which has no variables.
%
Combinatory completeness says that terms can be internalized in a pca. Indeed,
we can represent \(t\) in \(\AA\) as \(\lambdapca{xyz}{t}\). Note that this is
indeed an element of \(\AA\) by virtue
of~\cref{exer:properties-of-lambda-pca}\ref{lambda-pca-def} and the fact that
\(\lambdapca{xyz}{t}\) is a closed term.

% \begin{theorem}[Combinatory completeness]
%   For every term \(t\) with variables \(x_1,\dots,x_{n+1}\), there is an element
%   \(\pca{a} \in \AA\) such that for all
%   \(\pca{b_1},\dots,\pca{b_{n+1}} \in \AA\), we have:
%   \begin{enumerate}[(i)]
%   \item \(\pca{a}\pca{b_1}\cdots\pca{b_n}\) is defined, and
%   \item \(\pca{a}\pca{b_1}\cdots\pca{b_n}\pca{b_{n+1}}\)
%   \end{enumerate}
% \end{theorem}
% \begin{proof}
% \end{proof}

\begin{remark}
  While similar to \(\lambda\)-abstraction, we deliberately do not use the
  \(\lambda\)-symbol, because it might suggest that \(\lambdapca{x}{t}\) obeys the
  \emph{\(\beta\)-law} when substituting terms, but due to the partial nature of
  the application map, this need not be the case~\cite[p.~4]{vanOosten2008}.
\end{remark}

It is now time to make use of combinatory completeness to construct some more
combinators. We will do this in very much the same way as one encodes these
constructs in the untyped \(\lambda\)-calculus.

\subsubsection*{Booleans and conditional}\label{sec:booleans}

We define the booleans as the closed terms
\[
  \pcatrue \coloneqq \lambdapca{xy}{x} \quad\text{and}\quad \pcafalse \coloneqq
  \lambdapca{xy}{y},
\]
and the conditional as the closed term
\[
  \pcaif \coloneq \lambdapca{x}{x}.
\]
%
Note that for arbitrary \(\pca{a}\), \(\pca{b} \in \AA\), we can calculate:
\[
  \pcaif\pcatrue\pca{a}\pca{b} = \pcatrue\pca{a}\pca{b} = \pca{a}
  \quad
  \text{and}
  \quad
  \pcaif\pcafalse\pca{a}\pca{b} = \pcafalse\pca{a}\pca{b} = \pca{b}.
\]

\subsubsection*{Pairing and projection}

We define the closed terms
\[
  \pcapair \coloneqq \lambdapca{xyz}{z}{xy},
  \quad
  \pcafst \coloneqq \lambdapca{w}{w\pcatrue}
  \quad
  \text{and}
  \quad
  \pcasnd \coloneqq \lambdapca{w}{w\pcafalse}.
\]

\begin{exercise}\label{exer:pairing-projection}
  For all elements \(\pca{a}\), \(\pca{b} \in \AA\), prove that:
  \begin{enumerate}[(i)]
  \item \(\pcapair \pca{a} \pca{b}\) is defined.
  \item \(\pcafst{(\pcapair\pca{a}\pca{b})} = \pca{a}\) and
    \(\pcasnd{(\pcapair\pca{a}\pca{b})} = \pca{b}\) hold.
  \end{enumerate}
\end{exercise}

\subsubsection*{Fixed points}

We also have the fixed point combinators which are traditionally denoted by
\(\pcay\) and \(\pcaz\):
\begin{align*}
  \pcay &\coloneqq \pcaw\pcaw &\text{with}&& \pcaw &\coloneq \lambdapca{xy}{y\,(x\,x\,y)} \\
  \pcaz &\coloneqq \pcau\pcau &\text{with}&& \pcau &\coloneq \lambdapca{xyz}{y\,(x\,x\,y)\,z}
\end{align*}
These combinators satisfy:
\[
  \pcay\pca{f} \simeq \pca{f}{(\pcay\pca{f})},
    \quad
  \pcaz\pca{f} \text { is defined}
    \quad\text{and}\quad
  \pcaz\pca{f}\pca{a} \simeq \pca{f}{(\pcaz\pca{f})}\pca{a}
\]
for all elements \(\pca{f},\pca{a} \in \AA\).

\subsubsection*{Curry numerals and fundamental arithmetic}\label{sec:numerals}

For each natural number \(n \in \Nat\), we define the corresponding
\textbf{Curry numeral} \(\numeral{n}\) inductively by:
\[
  \numeral{0} \coloneq \icomb
  \quad\text{and}\quad
  \numeral{n+1} \coloneq \pcapair\pcafalse\numeral{n}.
\]

\begin{exercise}\label{exer:arithmetic}
  Define closed terms \(\pcasucc\), \(\pcapred\) and \(\pcaiszero\) such that
  for any natural number \(n \in \Nat\) the following equations hold:
  \begin{align*}
    \pcasucc\numeral{n} = \numeral{n+1},
    \quad
    \pcapred\numeral{0} = \numeral{0},
    \quad
    \pcapred\numeral{n+1} = \numeral{n}, \\
    %\quad
    \pcaiszero\numeral{0} = \pcatrue
    \quad\text{and}\quad
    \pcaiszero\numeral{n+1} = \pcafalse.
  \end{align*}

\end{exercise}

\subsubsection*{Primitive recursion}

The following \textbf{primitive recursion} combinator for Curry numerals will
come in useful when we consider the natural numbers object in the category of
assemblies later.

\begin{exercise}\label{exer:primitive-recursion}
  Construct a closed term \(\pcarec\) such that for all
  \(\pca{f},\pca{a} \in \AA\) and \(n \in \Nat\) it satisfies:
  \[
    \pcarec\pca{a}\pca{f}\numeral{0} = \pca{a}
    \quad\text{and}\quad
    \pcarec\pca{a}\pca{f}\numeral{n+1} \simeq \pca{f}\numeral{n}{(\pcarec\pca{a}\pca{f}\numeral{n})}
  \]
  \emph{Hint:} The essential ingredients are a zero test, predecessor and
  repeated application which are provided by \(\pcaiszero\), \(\pcapred\) and
  \(\pcaz\), respectively.
\end{exercise}

As a final remark, we note that any nontrivial pca is necessarily infinite, as
the following exercise shows:
\begin{exercise}\label{exer:nontrivial-pca}
  Prove that the following are equivalent for any pca \(\AA\):
  \begin{enumerate}[(i)]
  \item The booleans \(\pcatrue\) and \(\pcafalse\) are distinct elements of
    \(\AA\).
  \item The Curry numerals \(\numeral{n}\) in \(\AA\) are all distinct.
    % , i.e., the map
    % \(n \in \Nat \mapsto \numeral{n} \in \AA\) is an injection.
  \item The pca \(\AA\) is not the trivial pca.
  \end{enumerate}
\end{exercise}

\section{More examples of pcas}\label{sec:more-examples-of-pcas}

In this section we will present a few more examples of partial combinatory
algebras. To motivate and introduce them, we offer the following informal
explanation. Suppose we have a device which transforms the pitch of incoming
audio. We might represent the incoming and outgoing audio as streams of bits, so
that mathematically speaking, our in- and output are elements of
\(\set{0,1}^\Nat\), i.e.\ functions from the set of natural numbers \(\Nat\) to
the two-element set \(\set{0,1}\).
%
Our bit-stream transforming device is then a function
\(\set{0,1}^\Nat \to \set{0,1}^\Nat\).
%
Since we don't want to wait forever on the output of our device, it seems
reasonable that we assume it to start outputting after having only received
finitely many bits.
%
Thus, \emph{the output depends on a finite amount of the input only}.
%
Our device should therefore not be any old function
\(\set{0,1}^\Nat \to \set{0,1}^\Nat\), but rather one whose output depends on a
finite amount of input only.
%
This can be made mathematically precise by equipping the set \(\set{0,1}^\Nat\)
with a \emph{topology} and by restricting our attention to \emph{continuous}
functions on such topologized sets.

For an introduction to and motivation of topology from a computer science
perspective, Smyth's~\cite{Smyth1992} and Escard\'o's~\cite{Escardo2004}
monographs, and Vickers's book~\cite{Vickers1996} are warmly recommended.

While Turing computable functions, or equivalently, Kleene's partial recursive
functions, provide the foundation for computability theory with (encodings of)
finite objects, an established theory of computability over infinite objects is
Weihrauch's \emph{Type Two Effectivity (TTE)}~\cite{Weihrauch2000}.
%
This is a theory of computable analysis and becomes especially relevant when we
are interested in exact real number computation.
%
Its connections to realizability were explored in the PhD theses of
Lietz~\cite{Lietz2004} and Bauer~\cite{Bauer2000} via Kleene's second model and
its recursive variant (\cref{Kleene-2,Kleene-rec-2} below).

We will be brief in our explanations of these partial combinatory algebras and hope
that the interested reader will take the above paragraphs and the examples as an
invitation to explore the fascinating connections between topology and
computability, for instance by consulting Bauer's aforementioned notes on
realizability~\cite{Bauer2023} and the references above.

Our first example is due to Scott~\cite{Scott1976} and gives the powerset of the
natural numbers the structure of a pca.

\begin{example}[Scott's graph model, \(\Scott\)]
  We make the powerset \(\PN\) of the natural numbers into a pca that we denote
  by \(\Scott\).

  The idea behind application on \(\PN\) is that when we apply a subset \(U\) to
  a subset \(V\), then \(U\) can only use a finite amount of \(V\) to
  determine whether to include a number or not. We now make this idea formal.

  We first fix bijections
  \[
    \pairing{{-},{-}} \colon \Nat \times \Nat \to \Nat
    \quad\quad\text{and}\quad\quad
    \enum \colon \Nat \cong \powerset_{\operatorname{fin}}(\Nat)
  \]
  where the latter enumerates all finite subsets of natural numbers.
  %
  We then define the application map \(\PN \times \PN \to \PN\) as
  \[
    (U,V) \mapsto \set{\outputnum \in \Nat \mid
            \exists(\inputnum \in \Nat) .
            \pairing{\inputnum,\outputnum} \in U \text{ and } \enum(\inputnum) \subseteq V}.
  \]
  Thus, the ``program'' \(U\) encodes a list of input-output pairs where the
  input number encodes a finite subset of the argument \(V\).

  If we equip \(\PN\) with the \emph{Scott topology} whose basic opens are
  finite subsets of \(\Nat\), then one can show that the application map is
  continuous.
  %
  In fact, any continuous function \(F : \PN^k \to \PN\) can be represented in
  \(\PN\) by encoding the graph of \(F\).
  %
  (The details are spelled out in~\cite[Example~2.3.4]{deJong2018}.)
  %
  Under this correspondence it becomes easy to construct the elements \(\kcomb\)
  and \(\scomb\) making \(\PN\) into a pca.
\end{example}

The application in the above example is actually a \emph{total} operation. An
analogous pca whose application is genuinely partial is due to
Kleene~\cite{KleeneVesley1965}:
\begin{example}[Kleene's second model, \(\Ktwo\)]\label{Kleene-2}
  Somewhat similar to the above example, we can define an application on the set
  \(\Nat^\Nat\) of functions on \(\Nat\) that makes it into a pca~\(\Ktwo\),
  known as Kleene's second model.
  %
  The encodings required for the application map are a bit involved, see
  e.g.~\cite[p.~30 and Section~2.1.2]{Bauer2023} or
  \cite[Section~1.4.3]{vanOosten2008}, but we point out that this pca is closely
  related to continuous functions \(\Nat^\Nat \to \Nat^\Nat\) where we equip
  \(\Nat^\Nat\) with the \emph{Baire topology}.
\end{example}

Another example comes from \emph{domain theory}~\cite{AmadioCurien1998} and is
also due to Scott~\cite{Scott1972}:
\begin{example}[Scott's domain model of the untyped \(\lambda\)-calculus]
  The carrier of this pca is Scott's domain \(D_\infty\) which is a model of the
  untyped \(\lambda\)-calculus via the isomorphism
  \(\Phi \colon D_\infty \cong {D_\infty}^{D_\infty}\) of domains.
  %
  The map~\(\Phi\) sends an element \(\sigma \in D_\infty\) to a Scott
  continuous function \(\Phi(\sigma) \colon D_\infty \to D_\infty\).
  %
  We define application on \(D_\infty\) as:
  \[
    (\sigma,\tau) \mapsto \Phi(\sigma)(\tau).
  \]
\end{example}

Finally, we mention that Scott's graph model and Kleene's second model have
effective variations that are examples of \emph{elementary sub-pcas}.

\begin{definition}[Elementary sub-pca]\label{def:elementary-sub-pca}
  An \textbf{elementary sub-pca} of a pca \(\AA\) is a subset
  \(\AA^\# \subseteq \AA\) that is closed under the application of \(\AA\) and
  moreover contains the elements \(\kcomb\) and \(\scomb\) from \(\AA\).
\end{definition}

\begin{example}[Scott's r.e. graph model, \(\Scottre\)]\label{Scott-re}
  We get an elementary sub-pca \(\Scottre\) of~\(\Scott\) by restricting
  ourselves to the recursively enumerable subsets of \(\Nat\).
\end{example}

\begin{example}[Kleene's recursive second model, \(\Ktworec\)]\label{Kleene-rec-2}
  We get an elementary sub-pca \(\Ktworec\) of \(\Ktwo\) by restricting
  ourselves to the total recursive %(= Turing computable)
  functions
  \(\Nat \to \Nat\).
\end{example}

For even more examples of pcas,
see~\cite[Section~1.4]{vanOosten2008}.

\section{List of exercises}
\begin{enumerate}
\item \cref{exer:properties-of-lambda-pca}: On the fundamental properties of
  \(\lambdapca{x}{t}\).
\item \cref{exer:pairing-projection}: On the properties of the pairing and
  projection combinators.
\item \cref{exer:arithmetic}: On fundamental arithmetic in a pca.
\item \cref{exer:primitive-recursion}: On primitive recursion in a pca.
\item \cref{exer:nontrivial-pca}: On nontriviality of a pca.
\end{enumerate}




%%% Local Variables:
%%% mode: latexmk
%%% TeX-master: "../main"
%%% End:

\chapter{Categories of assemblies}\label{chap:assemblies}

In this chapter we describe the \emph{category of assemblies} over a fixed but
arbitrary pca.
%
In particular, we show that the category has many desirable properties, for
example it is (locally) cartesian closed and has finite colimits.
%
In fact, the category has sufficient structure to support an interpretation of
first order logic as we explore in~\cref{chap:logic}.

Roughly speaking, the category of assemblies has sets with computability
data as objects and computable functions between such sets as morphisms.
%, where the
%computability is with respect to the chosen pca.
%
The category of assemblies as a whole may be thought of as a world of
computable mathematics.

\begin{definition}[Assembly]
  An \textbf{assembly} over a pca \(\AA\) is a set \(X\) together with a
  relation \({\realizes}\) between \(\AA\) and \(X\) such that for all
  \(x \in X\), there exists at least one element \(\pca{a} \in \AA\) with
  \(\pca{a} \realizes x\).
\end{definition}

The relation \(\pca{a} \realizes x\) is pronounced as ``\(\pca{a}\)
\textbf{realizes} \(x\)'' and we also say that \(\pca{a}\) is a
\textbf{realizer} of \(x\). We think of \(\pca{a}\) as an \emph{implementation}
of \(x \in X\) in the pca.
%
The requirement on assemblies is that each element of the set must have at least
one implementation.

\begin{notation}[\(\carrier{X}\), \({\realizes_X}\)]
  Given an assembly \(X\), we will write \(\carrier{X}\) for its underlying set
  and \(\realizes_X\) for its relation between \(\AA\) and \(\carrier{X}\).
\end{notation}

\begin{example}[Assembly of booleans, \(\Two\)]\label{ex:assembly-of-booleans}
  The \textbf{assembly of booleans}, denoted by \(\Two\), is defined as
  \[
    \carrier{\Two} \coloneqq \set{0,1}
    \quad\text{with realizers}\quad
    \pcafalse \realizes_\Two 0
    %\quad
    \text{ and }%\quad
    \pcatrue \realizes_\Two 1,
  \]
  where we recall the booleans \(\pcafalse\) and \(\pcatrue\) from \cref{sec:booleans}.
\end{example}

\begin{example}[Assembly of natural numbers, \(\NatAsm\)]\label{ex:NatAsm}
  The \textbf{assembly of natural numbers}, denoted by \(\NatAsm\), is defined as
  \[
    \carrier{\NatAsm} \coloneqq \Nat
    \quad\quad\text{and}\quad\quad
    \numeral{n} \realizes_\NatAsm n \text{ for each \(n \in \Nat\)},
  \]
  where we recall from~\cref{sec:numerals} that \(\numeral{n}\) is the
  \(n\)\textsuperscript{th} Curry numeral.
\end{example}

\begin{example}
  Taking Kleene's first model as our pca, \(\AA = \Kone\), we can consider the
  assembly \(X\) of Turing computable functions:
  \[
    \carrier{X} \coloneqq \set{f \colon \Nat \to \Nat \mid f \text{ is Turing computable}}
    \quad\quad\text{and}\quad\quad
    m \realizes_X f \iff \prenum{m} = f
  \]
  where we recall \(\Kone\) and \(\prenum{-}\) from~\cref{ex:Kleene-1}.
  %
  We remark that each \(f \in \carrier{X}\) has infinitely many realizers.

  Notice that, with this realizability relation, we cannot let \(\carrier{X}\)
  be the set of \emph{all} functions from \(\Nat\) to \(\Nat\), because then the
  set of realizers of a noncomputable function would be empty, which is not
  allowed by the definition of an assembly.
\end{example}

In the literature, assemblies over \(\Kone\) are sometimes referred
to as \emph{\(\omega\)-sets}.

\section{Morphisms of assemblies}

As mentioned in the opening paragraphs of this chapter, we wish to organize the
assemblies into a category whose morphisms are ``computable'' functions between
the underlying sets of assemblies.
%
The computability requirement is made precise by requiring the existence of a
\emph{tracker} which we define now.

\begin{definition}[Track]
  For assemblies \(X\) and \(Y\), we say that an element \(\pca{t} \in \AA\)
  \textbf{tracks} a function \(f \colon \carrier{X} \to \carrier{Y}\) if for all
  \(x \in \carrier{X}\) and \(\pca{a} \in \AA\), if \(\pca{a} \realizes_X x\), then
  \(\pca{t}\pca{a}\) is defined and \(\pca{t}\pca{a} \realizes_Y f(x)\).
\end{definition}

\begin{notation}
  We will shorten the above to: ``\(\pca{t}\pca{a} \realizes_Y f(x)\)
  for all \(x \in \carrier{X}\) and \(\pca{a} \realizes_X x\)''.
  %
  That is, we implicitly quantify over \(\pca{a}\) and we implicitly assume that
  \(\pca{t}\pca{a}\) is defined when we write
  \(\pca{t}\pca{a} \realizes_Y f(x)\).
\end{notation}

\begin{definition}[Assembly map]
  An \textbf{assembly map} from an assembly \(X\) to an assembly \(Y\) is a
  function \(f \colon \carrier{X} \to \carrier{Y}\) that is tracked by some
  element.
\end{definition}

The existence of a tracker is a required \emph{property} of an assembly map and
\emph{not} part of the data, so an assembly map is (unlike an assembly)
\emph{not} a pair of a function and a tracker; it is just a function for which
there exists some tracker.

\begin{proposition}
  Assemblies and assembly maps form a category with composition given
  by composition of functions on underlying sets.
\end{proposition}
\begin{proof}
  We need to verify that composition is well defined, i.e., that if
  \(f \colon X \to Y\) and \(g \colon Y \to Z\) are assembly maps, then
  \(g \circ f \colon \carrier{X} \to \carrier{Z}\) is tracked.
  %
  Let \(\pca{t_f}\) and \(\pca{t_g}\) track \(f\)~and~\(g\), respectively.  We
  claim that \(\lambdapca{x}{\pca{t_g}(\pca{t_f}(x))}\) tracks \(g \circ
  f\). Indeed, the closed term \(\lambdapca{x}{\pca{t_g}(\pca{t_f}(x))}\) is
  defined by construction, and if \(\pca{a} \realizes_X x\), then
  \[
    {(\lambdapca{x}{\pca{t_g}(\pca{t_f}(x)}))\pca{a} =
    \pca{t_g}(\pca{t_f}\pca{a})} \realizes_Z g(f(x))
  \]
  by choice of \(\pca{t_f}\) and \(\pca{t_g}\).
  %
  Moreover, for each assembly \(X\), we have an identity morphism on \(X\) given
  by the identity on \(\carrier{X}\) and tracked by \(\icomb\).
  %
  Finally, associativity of composition holds because composing functions of
  sets is associative.
\end{proof}

\begin{notation}[\(\Asm{\AA}\)]
  We write \(\Asm{\AA}\) for the category of assemblies over a pca \(\AA\).
\end{notation}

\begin{example}
  For the trivial pca \(\AA = \set{\singleton}\) we recover the familiar
  category \(\Set\) of sets and functions as \(\Asm{\AA}\).
\end{example}

\begin{example}
  A paradigmatic example of a function that is \emph{not} tracked is obtained by
  taking \(\AA \coloneqq \Kone\) and considering the characteristic function of
  the \emph{Halting set}:
  \[
    f \colon \carrier{\NatAsm} \to \carrier{\Two}
    \quad\text{with}\quad
    f(n) \coloneqq 1 \text{ if \(\prenum{n}(n)\) is defined}
    \quad\text{and}\quad
    f(n) \coloneqq 0 \text{ otherwise}.
  \]
  The existence of a tracker for this function says exactly that the Halting set
  is computable, which it (famously) isn't.
\end{example}

\begin{remark}[Relative categories of assemblies]
  An interesting variation on the category of assemblies is obtained if we start
  with a pca \(\AA\) and an elementary sub-pca \(\AA^\#\)
  (recall~\cref{def:elementary-sub-pca} and \cref{Scott-re,Kleene-rec-2}).
  %
  While we still ask that the realizers of elements of sets come from the
  larger pca \(\AA\), we now require the trackers of assembly maps to come from
  the smaller sub-pca \(\AA^\#\) instead.
  %
  The requirements on an elementary sub-pca guarantee that this is again a
  category which we call the \textbf{relative category of assemblies}.
  %and denote
  %by \(\Asm{\AA,\AA^\#}\).

  Especially in the typical examples (\ref{Scott-re} and \ref{Kleene-rec-2})
  where \(\AA = \Ktwo\) and \(\AA^\# = \Ktworec\), or \(\AA = \Scott\) and
  \(\AA^\# = \Scottre\), the idea of the relative categories of assemblies is
  nicely captured by Bauer's~\cite[p.~36 and 45]{Bauer2000,Bauer2023}
  slogan
  \begin{center}
    \emph{Topological data --- computable functions!}
  \end{center}
  %nicely captures the intent of the relative categories of assemblies.
\end{remark}

\section{Categorical constructions}

This section shows that the category of assemblies has a rich categorical
structure. In particular, we will construct finite (co)products, exponentials
(in slices) and (co)equalizers of assemblies.
%
This structure is important for the interpretation of first order logic in
assemblies (\cref{chap:logic}).%that we explore further in \cref{chap:logic}.
%
But besides this, presenting the required constructions provides excellent
opportunities for improving our understanding of the category of assemblies.

\subsection{Cartesian closure and equalizers}
We start by showing that the category of assemblies is cartesian closed and has
equalizers. The description of the latter will prove useful in our study of
regular monomorphisms of assemblies (\cref{sec:regular-monos}).

\begin{proposition}[Terminal object]
  The terminal object \(\One\) in \(\Asm{\AA}\) is given by
  \[
    \carrier{\One} \coloneq \set{\singleton}
    \quad
    \text{and}
    \quad
    \pca{a} \realizes_\One \singleton
    \text{ for all } \pca{a} \in \AA.
  \]
\end{proposition}
\begin{proof}
  As in \(\Set\).
\end{proof}

The pairing and projection combinators from~\cref{sec:basic-programming-in-pcas}
are essential in the construction of finite products of assemblies.

\begin{proposition}[Products]
  The product \(X \times Y\) of two assemblies \(X\)~and~\(Y\) is given by
  \[
    \carrier{X \times Y} \coloneq \carrier{X} \times \carrier{Y}
    \quad\text{and}\quad
    \pcapair\pca{a}\pca{b} \realizes_{X \times Y} (x,y)
    \text{ for }
    \pca{a} \realizes_X x
    \text{ and }
    \pca{b} \realizes_Y y.
  \]
\end{proposition}
\begin{proof}
  The projection maps \(\pi_1 \colon X \times Y \to X\) and
  \(\pi_2 \colon X \times Y \to Y\) are given by \((x,y) \mapsto x\) and
  \((x,y) \mapsto y\), and tracked by \(\pcafst\) and \(\pcasnd\), respectively.
  %
  Moreover, every pair of assembly maps \(f \colon Z \to X\) and
  \(g \colon Z \to Y\) induces an assembly map
  \(\langle f,g\rangle \colon Z \to X \times Y\) given by
  \(z \mapsto (f(z),g(z))\) and tracked by
  \(\lambdapca{u}{\pcapair(\pca{t_f}u)(\pca{t_g}u)}\) when \(\pca{t_f}\) and
  \(\pca{t_g}\) track \(f\) and \(g\), respectively.
\end{proof}

So far, the underlying sets of the terminal object and product of two assemblies
have been exactly as in \(\Set\), e.g. \(\carrier{X \times Y}\) is the product
of the two sets \(\carrier{X}\) and \(\carrier{Y}\) in \(\Set\).
%
A notable exception to this is the exponential (a.k.a.\ ``internal hom''):
the object of morphisms between two assemblies.
%
It would not make sense for the carrier of the exponential of assemblies \(X\)
and \(Y\) to consist of all functions from \(\carrier{X}\) to \(\carrier{Y}\),
because (a) the exponential is usually similar to the hom-set of morphisms from
\(X\) to \(Y\), and (b) it would not be clear what the realizers of an arbitrary
function between carriers should be.
%
Instead, the exponential is given by assembly maps only.

\begin{proposition}[Exponentials]
  The exponential \(Y^X\) of two assemblies \(X\)~and~\(Y\) is given by
  \[
    \carrier*{Y^X} \coloneq
    \text{the set of assembly maps from \(X\) to \(Y\)}
    \quad\text{and}\quad
    \pca{t} \realizes_{Y^X} f
    \text{ if \(\pca{t}\) tracks \(f\)}.
  \]
\end{proposition}
\begin{proof}
  The evaluation morphism \(\operatorname{ev} \colon Y^X \times X \to Y\) given
  by \((f,x) \mapsto f(x)\) is tracked by
  \(\lambdapca{u}{\pcafst u(\pcasnd u)}\).
  %
  Moreover, every \(g \colon Z \times X \to Y\) induces a unique assembly map
  \(\tilde g \colon Z \to Y^X\) making the diagram
  \[
    \begin{tikzcd}
      Y^X \times X \ar[rr,"\operatorname{ev}"]
      & & Y \\
      & Z \times X \ar[ur,"g"']
      \ar[ul,"{\tilde g} \,\times\, {\id_X}"]
    \end{tikzcd}
  \]
  commute.
  %
  Indeed, there is a unique assignment
  \(\tilde g(z) \coloneq (x \mapsto g(z,x))\) and this assignment is tracked by
  \(\lambdapca{u}{(\lambdapca{v}{\pca{t_g}(\pcapair u\,v)})}\) when
  \(\pca{t_g}\) tracks \(g\).
\end{proof}

Thus, we conclude that the category \(\Asm{\AA}\) is cartesian closed.

Finally, we construct equalizers in \(\Asm{\AA}\). Their description will come
in useful when we study regular monomorphisms in \cref{sec:regular-monos}. We
also note that we obtain an explicit construction of pullbacks when combined
with the construction of products.

\begin{proposition}[Equalizers]
  The equalizer \(E\) of two assembly maps \(f,g \colon X \to Y\) is given by
  \[
    \carrier{E} \coloneq \set{x \in \carrier{X} \mid f(x) = g(x)}
    \quad\text{and}\quad
    \pca{a} \realizes_E x \text{ if } \pca{a} \realizes_X x.
  \]
\end{proposition}
\begin{proof}
  On the level of underlying sets, this is as in the category of sets, so it
  suffices to show that the relevant functions are tracked.
  %
  The inclusion \(i \colon \carrier{E} \to \carrier{X}\) is tracked by
  \(\icomb\).
  %
  Given an assembly map \(h \colon D \to X\) such that
  \(f \circ h = g \circ h\), the map \(h \colon \carrier{D} \to \carrier{X}\)
  factors uniquely through \(i\) via \(k \colon \carrier{D} \to \carrier{E}\)
  which is tracked by any tracker of \(h\).
\end{proof}

\subsection{Coproducts and coequalizers}
%
We now move on to colimits in the category of assemblies.

\begin{proposition}[Initial object]
  The initial object \(\Zero\) in \(\Asm{\AA}\) is given by
  \( \carrier{\Zero} \coloneq \emptyset \) with the empty realizability
  relation.
\end{proposition}
\begin{proof}
  As in \(\Set\).
\end{proof}

\begin{proposition}[Coproducts]\label{coproducts}
  The coproduct \(X + Y\) of two assemblies \(X\) and \(Y\) is given by
  \begin{align*}
    \carrier{X + Y} \coloneq \carrier{X} + \carrier{Y}
    \quad\text{and}\quad
    \pcaleft\pca{a} &\realizes_{X + Y} \inl(x)
      \text{ for } \pca{a} \realizes_X x,
    \\
    \pcaright\pca{b} &\realizes_{X + Y} \inr(y)
    \text{ for }
    \pca{b} \realizes_Y y,
  \end{align*}
  where
  \[
    \pcaleft \coloneqq \pcapair\pcafalse
    \text{ and }
    \pcaright \coloneqq \pcapair\pcatrue.
  \]
\end{proposition}

Notice that the disjointness of the coproduct is witnessed by tagging the
realizers with booleans.

\begin{exercise}\label{exer:coproducts}
  Prove~\cref{coproducts}.

  \emph{Warning}: Carefully check that the closed terms you give for the
  trackers are defined and thus give elements of the pca as required.
\end{exercise}

While there are several interesting assemblies whose carrier is the set
\(\set{0,1}\) (as we will see in \cref{chap:logic}), the following exercise
justifies the notation \(\Two\) for the assembly of booleans as defined in
\cref{ex:assembly-of-booleans}.

\begin{exercise}\label{exer:coproduct-booleans}
  Show that \({\One + \One} \cong \Two\).
\end{exercise}

Finally, we construct coequalizers in \(\Asm{\AA}\). Their description will be
useful in our study of regular epimorphisms (\cref{sec:regular-epis}).

\begin{proposition}[Coequalizers]\label{prop:coequalizers}
  The coequalizer \(C\) of assembly maps \(f,g \colon X \to Y\) is given by
  \[
    \carrier{C} \coloneq \carrier{Y}/{\sim}
    \quad\text{and}\quad
    \pca{a} \realizes_C [y] \text{ if } \pca{a} \realizes_Y y'
    \text{ for some } {y'} \sim y,
  \]
  where \({\sim}\) is the least equivalence relation on \(\carrier{Y}\)
  generated by \(f(x) \sim g(x)\) for all \(x \in \carrier{X}\).
  % setting \(y \sim y'\) whenever there exists \(x \in \carrier{X}\)
  % with \(f(x) = y\) and \(g(x) = y'\).
\end{proposition}
\begin{proof}
  On the level of underlying sets, this is as in the category of sets, so it
  suffices to show that the relevant functions are tracked.
  %
  The quotient map \(q \colon \carrier{Y} \to \carrier{Y}/{\sim}\) is tracked by
  \(\icomb\).
  %
  Given an assembly map \(h \colon Y \to Z\) such that
  \(h \circ f = h \circ g\), the map \(h \colon \carrier{Y} \to \carrier{Z}\)
  factors uniquely through \(q\) via
  \(k \colon \carrier{Y}/{\sim} \to \carrier{Z}\) with \(k([y]) \coloneq h(y)\).
  %
  Moreover, the function \(k\) is tracked by any tracker of \(h\).
\end{proof}

\subsection{Natural numbers object}
The notion of natural numbers can be captured categorically via a universal
property which is due to Lawvere~\cite{Lawvere1963}. In a category
\(\mathcal C\), a \textbf{natural numbers object (nno)} is an object \(N\)
equipped with morphisms \(z \colon 1 \to N\) (``zero'') and \(s \colon N \to N\)
(``successor'') such that for all triples
\((X,x \colon 1 \to X,f\colon X \to X)\) there is a unique morphism
\(r \colon N \to X\) (defined by ``recursion'') making the diagram
\[
  \begin{tikzcd}[row sep=8mm,column sep=8mm]
    & N \ar[r,"s"] \ar[d,dashed,"r"] & N \ar[d,dashed,"r"] \\
    1 \ar[ur,"z"] \ar[r,"x"] & X \ar[r,"f"] & X
  \end{tikzcd}
\]
commute.

\begin{exercise}\label{exer:nno} \leavevmode
  \begin{enumerate}[(i)]
  \item Exhibit \(\Nat\) as a nno in \(\Set\).
  \item Exhibit \(\NatAsm\) (from~\cref{ex:NatAsm}) as a nno in \(\Asm{\AA}\).

    \emph{Hint}: Use~\cref{exer:primitive-recursion}.
  \end{enumerate}
\end{exercise}

Looking ahead to \cref{chap:logic} we note that if we wish to interpret
arithmetic in the category of assemblies, then its natural numbers object serves
as the interpretation of the sort of natural numbers.

\subsection{Dependent products and sums}

For the purposes of these lecture notes, a proof sketch of the following result
suffices. We discuss the significance of the result below.

\begin{proposition}\label{base-change-adjoints}
  For every morphism \(f \colon X \to Y\) of assemblies, the pullback functor
  \(f^\ast \colon \Asm{\AA}/Y \to \Asm{\AA}/X\) has both a left adjoint
  \(\sum_f\) and a right adjoint \(\prod_f\).
\end{proposition}
\begin{proof}[Proof sketch]
  We only describe the constructions and leave the verification of the details
  to the interested reader.
  %
  The left adjoint \(\sum_f\) takes an object \(g \colon Z \to X\) of
  \(\Asm{\AA}/X\) to the object \(f \circ g\) of \(\Asm{\AA}/Y\).
  %
  On morphisms it is the identity.

  The right adjoint \(\prod_f\) is more involved.
  %
  Given an object \(g \colon Z \to X\) of \(\Asm{\AA}/X\), we consider the
  assembly \(P\) of ``fiberwise maps''. It is given by
  \[
    \carrier{P} \coloneq
    \set{(y,s) \mid s \colon f^{-1}(y) \to Z \text{ such that }
      \forall(x \in \carrier{f^{-1}(y)})\,.\,s(x) \in \carrier{g^{-1}(x)}},
  \]
  where
  \[
    \carrier{f^{-1}(y)} \coloneq \set{x \in \carrier{X} \mid f(x) = y}
    \quad\text{with realizers}\quad
    \pca{a} \realizes_{f^{-1}(y)} x \text{ if } \pca{a} \realizes_X x,
  \]
  (and similarly for \(g^{-1}(x)\)), and for realizers, we put
  \[
    \pcapair\pca{b}\pca{t} \realizes_P (y,s)
    \text{ if }
    \pca{b} \realizes_Y y
    \text{ and}
    \pca{t} \text{tracks } s.
  \]
  Now, \(P\) defines an object of \(\Asm{\AA}/Y\) by considering the first
  projection \(\pi_1 \colon P \to Y\) which is tracked by \(\pcafst\).
  %
  This projection map is the value of \(\prod_f(g)\).
  %
  Given a morphism
  \[
    \begin{tikzcd}[row sep=4mm,column sep=4mm]
      Z \ar[rr,"k"] \ar[dr,"g"'] & & W \ar[dl,"h"] \\
      & X
    \end{tikzcd}
  \]
  in \(\Asm{\AA}/X\), we define \(\prod_f(k) \colon \prod_f(g) \to \prod_f(h)\)
  as a function on sets by \((y,s) \mapsto (y, k \circ s)\). This assignment can
  be shown to be tracked because \(k\) and \(s\) are.
\end{proof}

The above proposition is equivalent to the fact that the category \(\Asm{\AA}\)
is \emph{locally cartesian closed}, i.e.\ that each slice category
\(\Asm{\AA}/X\) is cartesian closed.
%
One may check that the exponential \(g^f\) of two objects
\(f,g \in \Asm{\AA}/X\) is given by \(\prod_f(f^\ast(g))\).

The functors \(\sum_f\) and \(\prod_f\) also satisfy the so-called
Beck--Chevalley condition~\cite[Theorem~4.4]{Streicher2018}, which we don't
spell out here. Intuitively, this condition expresses that
\(\prod\)~and~\(\sum\) preserve substitution which is given by pullback (see
e.g.~\cite{Bauer2012} for an informal explanation of the latter).

The adjoints allow us to interpret Martin-L\"of dependent type theory, where, as
the notation suggests, dependent sums and products are interpreted using
\(\sum\) and \(\prod\), respectively. The interested reader may
consult~\cite{Seely1984,Jacobs1999,Streicher1991} to learn more.
%
We implicitly rely on (a slight variation of) \cref{base-change-adjoints} for
the interpretation of first order logic in the category of assemblies in
\cref{chap:logic}, where we only need the adjoints for \(f\) a projection
map.
%
But our presentation in \cref{chap:logic} will spell things out in more concrete
terms, so understanding \cref{base-change-adjoints} in detail is not strictly
required.

\section{Relation to the category of sets}\label{sec:relation-to-Set}

We introduce an adjunction
\[
\begin{tikzcd}
\mathbb{\Asm{\AA}}
\arrow[r, "\Gamma"{name=F}, shift left=2mm] &
\mathbb{\Set}
\arrow[l, "\nabla"{name=G}, shift left=2mm]
\arrow[phantom, from=F, to=G, "\dashv" rotate=-90]
\end{tikzcd}
\]
relating the categories of sets and assemblies.

\begin{definition}[Forgetful functor, \(\Gamma\)]
  The \textbf{forgetful functor}
  \[
    \Gamma \colon \Asm{\AA} \to \Set
  \]
  is defined by taking an assembly \(X\) to its underlying set \(\carrier{X}\)
  and a morphism \(f \colon X \to Y\) of assemblies to the map of sets
  \(f \colon \carrier{X} \to \carrier{Y}\).
\end{definition}

\begin{exercise}\label{exer:Gamma-global-sections}
  Show that \(\Gamma\) is naturally isomorphic to the \textbf{global sections}
  functor:
  \begin{align*}
    \Asm{\AA} &\to \Set \\
    X &\mapsto \Asm{\AA}(\One,X) \\
    f \colon X \to Y &\mapsto \text{post-composition with \(f\)}
  \end{align*}
\end{exercise}

To get a functor \(\nabla \colon \Set \to \Asm{\AA}\) we need a procedure for
turning a set \(X\) into an assembly \(\nabla X\).
%
Of course, it makes sense to let \(\carrier{\nabla(X)}\) be the set \(X\), but
what should the realizers be?
%
In general, there is no reason why an element \(x\) of an arbitary set \(X\)
should have an ``implemention'' in the pca \(\AA\).
%
In light of this, one might be tempted to say that there are no realizers of
\(x\). This is not a good idea for two reasons:
\begin{enumerate}[(1)]
\item It does not define an assembly, because, by definition, every element should
  have at least one realizer.
\item Even if the definition of assembly did allow for an empty set of
  realizers, then the tracking requirement on assembly maps would mean that
  there are no assembly maps \(\Two \to \nabla\set{0,1}\) which does not make
  sense if \(\nabla\set{0,1}\) is an assembly with no computational data.
\end{enumerate}

The solution is to say that \emph{all} elements of \(\AA\) are actually realizers
of \(x \in \carrier{\nabla(X)}\).
%
The idea is that \(\nabla(X)\) carries no meaningful computational data, because
all elements have the same set of realizers, so from this perspective all
elements of the assembly look alike and computationally speaking we can't tell
them apart.


\begin{definition}[\(\nabla\)]
  We define a functor
  \[
    \nabla \colon \Set \to \Asm{\AA}
  \]
  by mapping a set \(X\) to the assembly with carrier \(X\) and
  \(\pca{a} \realizes_X x\) for \emph{all} elements \(\pca{a} \in \AA\).
  %
  A map of sets \(f \colon X \to Y\) gets sent to \(f\) and is tracked by
  \(\icomb\).
\end{definition}

\begin{exercise}\label{exer:Gamma-left-adjoint-to-nabla}
  Prove that \(\Gamma\) is left adjoint to \(\nabla\).
\end{exercise}

\begin{notation}[\(\eta\)]
  We write \(\eta\) for the unit of the adjunction \(\Gamma \dashv \nabla\),
  i.e.\ for each assembly \(X\), we have an assembly map
  \[
    \eta_X \colon X \to \nabla \carrier{X}
  \]
  given by the identity on \(\carrier{X}\) and tracked by (for example)
  \(\icomb\).
\end{notation}

In particular, we have an assembly map
\(\eta_\Two \colon \Two \to \nabla\set{0,1}\).
%
We don't expect a non-constant map in the reverse direction as its tracker would
compute the relevant boolean (\(\pcatrue\) or \(\pcafalse\)) without receiving
any relevant computational input. Indeed, we have:

\begin{exercise}\label{exer:no-nabla-to-Two}
  Show that there are no non-constant assembly maps
  \(f \colon \nabla\set{0,1} \to \Two\) in \(\Asm{\AA}\) unless the pca \(\AA\)
  is trivial.

  \emph{Hint}: Use~\cref{exer:nontrivial-pca}.
\end{exercise}

To complete the section, the following exercises ask you to check that the
functors \(\Gamma\) and \(\nabla\) do not have other adjoints.

\begin{exercise}\label{exer:nabla-no-right-adjoint}
  Show that \(\nabla\) does not have a right adjoint when \(\AA\) is nontrivial.

  % \emph{Hint}: Consider the assemblies \(\Two\) (recall~\cref{TODO}) and \(\nabla\set{0,1}\).
\end{exercise}

\begin{exercise}[cf.~{\cite[Lemma~5.1.7]{Zoethout2018}}]\label{exer:Gamma-no-left-adjoint}
  We assume that the pca \(\AA\) is nontrivial. The aim of these exercises is to
  conclude that \(\Gamma\) does not have a left adjoint.
  \begin{enumerate}[(i)]
  \item Show that, for any object \(X \in \Asm{\AA}\), there are at most
    \(\carrier{\AA}\)-many arrows from \(X\) to \(\Two\).
  \item Use the above to prove that the \(\AA\)-indexed coproduct of copies of
    \(\One\) does not exist in \(\Asm{\AA}\).
  \item Conclude that \(\Gamma\) does not have a left adjoint.
  \end{enumerate}
\end{exercise}

The functor \(\nabla\) plays an important role in classifying regular
monomorphisms and recovering classical logic within realizability logic as
explained in \cref{chap:logic}.

\section{Epimorphisms and monomorphisms}
Epimorphisms and monomorphisms and their regular counterparts will be very
important to the logical side of the category of assemblies (see
\cref{chap:logic}), but studying them also provides excellent opportunities
for improving our understanding and intuition of assemblies in general.

\begin{proposition}[Characterization of epis and monos]\label{prop:characterize-epis-and-monos}
  For an assembly map \(f\), we have the following equivalences:
  \begin{enumerate}[(i)]
  \item \(f \colon X \to Y\) is an epimorphism if and only if
    \(f \colon \carrier{X} \to \carrier{Y}\) is surjective;
  \item \(f \colon X \to Y\) is a monomorphism if and only if
    \(f \colon \carrier{X} \to \carrier{Y}\) is injective.
  \end{enumerate}
\end{proposition}
\begin{proof}
  Surjectivity is clearly sufficient to force an assembly map to be epi.
  %
  For the converse, we use that \(\Gamma\) preserves epimorphisms as it is a
  left adjoint%
  \footnote{In any category, a morphism \(f\) is an epi if and only if the square
    \begin{tikzcd}[ampersand replacement=\&,column sep=3mm,row sep=3mm]
      X \ar[r,"f"] \ar[d,"f"'] \& Y \ar[d,"\id"] \\
      Y \ar[r,"\id"] \& Y
    \end{tikzcd}
    is a pushout. Since left adjoints preserve colimits (and identities), they
    also preserve epimorphisms\label{epi-mono-preservation}.}.
  %
  Thus, if \(f \colon \carrier{X} \to \carrier{Y}\) is an epi, then
  \(\Gamma(f)\) must be an epi in \(\Set\), i.e. a surjection.

  For the characterization of monomorphisms, injectivity is again clearly
  sufficient.
  %
  Conversely, it follows from our construction of products and equalizers that
  \(\Gamma\) preserves monomorphisms (use the dual
  of~\footref{epi-mono-preservation}).
  % \footnote{Use the dual
  %  of~\footref{epi-mono-preservation}.}.
  %
  Thus, if \(f \colon \carrier{X} \to \carrier{Y}\) is a mono, then
  \(\Gamma(f)\) must be a mono in \(\Set\), i.e. an injection.
\end{proof}

At first sight, it is perhaps surprising that being an epimorphism or
monomorphism depends solely on the underlying function of an assembly map with
the realizers playing no role.
%
The situation is very different for \emph{regular} epimorphisms and
monomorphisms.

\subsection{Regular epimorphisms}\label{sec:regular-epis}
Recall that a morphism \(f \colon X \to Y\) is a \textbf{regular epimorphism}
if it fits in a coequalizer diagram
\[
  \begin{tikzcd}
    Z \ar[r,shift left, "g"]\ar[r, shift right, "h"']
    & X \ar[r,"f"]
    & Y
  \end{tikzcd}
\]
for some morphisms \(g\) and \(h\).


\begin{exercise}[Characterization of regular epimorphisms]%
  \label{exer:characterize-regular-epis}
  Prove that an assembly map \(f \colon X \to Y\) is a regular epimorphism if
  and only if \(f \colon \carrier{X} \to \carrier{Y}\) is surjective and there
  exists an element \(\pca{s} \in \AA\) such that for all \(y \in \carrier{Y}\)
  and \(b \realizes_Y y\), we have \(\pca{s}\pca{b} \realizes_X x\) for some
  \(x \in \carrier{X}\) with \(f(x) = y\).
\end{exercise}

The element \(\pca{s}\) in \cref{exer:characterize-regular-epis} effectively
witnesses the surjectivity of \(f\).

\begin{exercise}\label{exer:epi-but-not-regular-epi}
  Give an example of an epimorphism in \(\Asm{\AA}\) which is not regular (for a
  nontrivial pca \(\AA\)).
\end{exercise}

\begin{proposition}\label{regular-epi-pullback-stable}
  The regular epimorphisms in \(\Asm{\AA}\) are stable under pullback along
  arbitrary assembly maps.
\end{proposition}
\begin{proof}
  Consider a pullback diagram
  \[
    \begin{tikzcd}
      X \times_Z Y \pbcorner
      \ar[r,"\pi_2"]
      \ar[d,"\pi_1"']
      & Y \ar[d,"g"] \\
      X\ar[r,"f"] & Z
    \end{tikzcd}
  \]
  with \(g\) a regular epimorphism. We must show that \(\pi_1\) is also a
  regular epi.
  %
  From the description of equalizers and products we can compute that
  \begin{align*}
    &\carrier{X \times_Z Y} \coloneq \set{(x,y) \mid f(x) = g(y)}
    \text{ with realizers}\\
    &\pcapair\pca{a}\pca{b} \realizes_{X \times_Z Y} (x,y)
    \text{ for }
    \pca{a} \realizes_X x
    \text{ and }
    \pca{b} \realizes_Y y.
  \end{align*}
  By assumption and~\cref{exer:characterize-regular-epis} there exists an
  element \(s \in \AA\) such that for every \(z \in \carrier{Z}\) and
  \(\pca{c} \realizes_Z z\) we have \(\pca{s}\pca{c} \realizes_Y y\) for some
  \(y \in \carrier{Y}\) with \(g(y) = z\).
  %
  Now if \(\pca{t}\) tracks \(f\) and we put
  \[
    \pca{s'} \coloneq \lambdapca{u}{\pcapair u\,(\pca{s}(\pca{t}u))},
  \]
  then for every \(x \in \carrier{X}\) and \(\pca{a} \realizes_X x\) we have
  \(\pca{s'}\pca{a} \realizes_{X \times_Z Y} (x,y)\) for some
  \(y \in \carrier{Y}\) with \(g(y) = f(x)\).
  %
  Hence, \(\pi_1\) is a regular epi by~\cref{exer:characterize-regular-epis}, as
  desired.
\end{proof}

The importance of \cref{regular-epi-pullback-stable} is that, together with
\cref{base-change-adjoints}, it gives the category of assemblies enough
structure to interpret first order logic as we explore in the next chapter.

We recall that in the category of sets, every function \(f \colon A \to B\)
factors as a regular epimorphism (= surjection) followed by a monomorphism (=
injection):
\[
  \begin{tikzcd}[row sep=3mm,column sep=3mm]
    A \ar[dr,"\tilde f"'] \ar[rr,"f"] & & B \\
    & \im(f) \ar[ur,hookrightarrow]
  \end{tikzcd}
\]
where \(\im(f) \coloneq \set{b \in B \mid \exists (a \in A).f(a) = b}\).
%

The same is true in any category with finite limits and pullback stable regular
epimorphisms.
%
For the category of assemblies it is also straightforward to calculate the
factorization directly, as we ask you to verify:

\begin{exercise}\label{exer:reg-epi-mono-factorization}
  Given an assembly map \(f \colon X \to Y\), show how to factor it in
  \(\Asm{\AA}\) as a regular epimorphism followed by a monomorphism.
\end{exercise}

\subsection{Regular monomorphisms}\label{sec:regular-monos}
There is also a nice characterization of regular monomorphisms which,
as we explain in the next chapter, represent classical (or more precisely,
\(\lnot\lnot\)-stable) subsets of assemblies.

Recall that a morphism \(f \colon X \to Y\) is a \textbf{regular monomorphism}
if it fits in an equalizer diagram
\[
  \begin{tikzcd}
    X \ar[r,"f"]
    & Y \ar[r,shift left, "g"]\ar[r, shift right, "h"']
    & Z
  \end{tikzcd}
\]
for some morphisms \(g\) and \(h\).

\begin{exercise}[Characterization of regular monomorphisms]%
  \label{exer:characterize-regular-monos}
  Prove that an assembly map \(f \colon X \to Y\) is a regular monomorphism if
  and only if \(f \colon \carrier{X} \to \carrier{Y}\) is injective and there
  exists an element \(\pca{i} \in \AA\) such that
  \(\pca{i}\pca{b} \realizes_X x\) for all \(x \in \carrier{X}\) and
  \(\pca{b} \realizes_Y f(x)\).
\end{exercise}

Notice that the element \(\pca{i}\) in \cref{exer:characterize-regular-monos}
acts like an effective witness of left-cancellability of \(f\): from a realizer
of \(f(x)\) we can effectively find a realizer of \(x\).

\begin{exercise}\label{exer:mono-but-not-regular-mono}
  Give an example of a monomorphism in \(\Asm{\AA}\) which is not regular (for a
  nontrivial pca \(\AA\)).
\end{exercise}


\section{From pcas to assemblies, functorially}
In this chapter we constructed a category of assemblies \(\Asm{\AA}\) over an
arbitrary pca \(\AA\). The construction \(\AA \mapsto \Asm{\AA}\) is actually
functorial in a suitable sense.
%
To make this precise, we need a \emph{category} of pcas.
%
On first thought, one might think that a suitable notion of a morphism between
pcas \(\AA\) and \(\BB\) is a map on the underlying sets of the pcas that
preserves application.
%
This turns out \emph{not} to be an appropriate notion of morphism however. It
would be if a pca should be thought of as an algebraic gadget, but it shouldn't:
for example, the application map is associative if and only if the pca is
trivial~\cite[Proposition~1.3.1]{vanOosten2008}.
%

Instead, an appropriate notion of morphism, due to Longley~\cite{Longley1995},
is that of an \emph{applicative morphism} (re-branded to \emph{simulation}
in~\cite{Bauer2023}).
%
For lack of space, we will not go into the details here and instead give some
relevant pointers to the literature.

The intuition is that a morphism between pcas from \(\AA\) to \(\BB\) should
assign to each program \(\pca{a} \in \AA\) one or more programs in \(\BB\) that
\emph{simulate} \(\pca{a}\) in \(\BB\).
%
Moreover, similar to the requirement that assembly maps are tracked, the
simulation should have an effective witness in the pca \(\BB\).
%
With applicative morphisms between them we get a category of pcas. In fact, this
category is preorder-enriched, so we even get a 2-category.

Now, one can show that every applicative morphism \(\gamma \colon \AA \to \BB\)
gives rise to a functor \(F_\gamma \colon \Asm{\AA} \to \Asm{\BB}\).
%
The induced functor \(F_\gamma\) is regular (i.e.\ it preserves finite limits
and regular epis) and is a so-called \emph{\(S\)-functor}: it is the
identity on underlying sets and functions.
%
In fact, every such functor arises from an applicative morphism.
%
In the end, we actually get an equivalence of
2-categories~\cite[Theorem~1.6.2]{vanOosten2008} between:
\begin{itemize}
\item the 2-category of pcas with applicative morphisms and their preorders, and
\item the 2-category of categories of assemblies with regular \(S\)-functors between
  them and (necessarily unique) natural transformations between the functors.
\end{itemize}




\section{List of exercises}
\begin{enumerate}
\item \cref{exer:coproducts}: On the universal property of coproducts.
\item \cref{exer:coproduct-booleans}: On the assembly of booleans as the
  coproduct \(\One + \One\).
\item \cref{exer:nno}: On natural numbers objects.
\item \cref{exer:Gamma-global-sections}: On the forgetful functor \(\Gamma\) and
  the global sections functor.
\item \cref{exer:Gamma-global-sections}: On \(\Gamma\) being a left adjoint to
  \(\nabla\).
\item \cref{exer:no-nabla-to-Two}: On the nonexistence of assembly maps
  \(\nabla\set{0,1} \to \Two\).
\item \cref{exer:nabla-no-right-adjoint}: On the nonexistence of a right
  adjoint to \(\nabla\).
\item \cref{exer:Gamma-no-left-adjoint}: On the nonexistence of a left
  adjoint to \(\Gamma\).
\item \cref{exer:characterize-regular-epis}: On characterizing the regular
  epimorphisms.
\item \cref{exer:epi-but-not-regular-epi}: On an example of an epimorphism
  which is not a regular.
\item \cref{exer:reg-epi-mono-factorization}: On factoring an assembly map as a
  regular epi followed by a mono.
\item \cref{exer:characterize-regular-monos}: On characterizing the regular
  monomorphisms.
\item \cref{exer:mono-but-not-regular-mono}: On an example of a monomorphism
  which is not a regular.
\end{enumerate}

%%% Local Variables:
%%% mode: latexmk
%%% TeX-master: "../main"
%%% End:

\chapter{The realizability interpretation of logic}\label{chap:logic}

In this chapter we explore some of the logical aspects of the category of
assemblies over a pca.
%
% Assemblies over Kleene's first model (\cref{ex:Kleene-1}) are of particular
% interest to us as we explore concepts from computability theory, such as
% computably enumerable sets, from a logical perspective.
%
The theory of categorical logic informs us that any sufficiently structured
category supports an interpretation of logic.
%
% The amount of structure in the category determines the richness of the logic it
% supports.
For example, in a regular category we can interpret a fragment of
first order logic known as regular logic, while we can interpret higher order
logic in a topos.
%
We include a brief introduction to categorical logic in
\cref{sec:categorical-logic}, but for a proper treatment we refer the reader to
the textbooks \cite{FreydScedrov1990} (first order logic),
\cite{LambekScott1986,MacLaneMoerdijk1994} (higher order logic), or the lecture
notes~\cite{vanOosten2016} (regular logic) and \cite{Streicher2004} (higher
order logic).
%
The key point is this:
\begin{quote}
  Categorical logic gives us a \emph{uniform} way of interpreting logic in any
  sufficiently structured category.
\end{quote}
%
Following Bauer's exposition in \cite{Bauer2023} we describe how to interpret
logic in the category of assemblies via the notion of a \emph{realizability
  predicate} on an assembly.
%
We emphasize that this is \emph{not} some ad-hoc interpretation. Rather, it is a
convenient unfolding of the uniform interpretation given to us by categorical
logic.
%
The structure established on the category of assemblies in
\cref{chap:assemblies} allows us to interpret full first order logic.

Having said all of this, the reader who finds themselves ill at ease with
categorical logic can find solace in the fact that the
particular---realizability---interpretation of logic is spelled out in more
elementary terms in \cref{sec:cat-logic-in-asm} and
\cref{sec:realizability-interpretation} in particular.

After establishing that the logic in the category of assemblies is given by
realizability predicates, \cref{sec:two-element-assemblies} isolates three
particular classes of such predicates by logical means. These are the
\(\lnot\lnot\)\nobreakdash-stable, decidable and semidecidable realizability
predicates.
%
In assemblies over Kleene's first model these are shown to respectively
correspond to ordinary (classical) subsets, computable subsets and computably
enumerable subsets.

Finally, \cref{sec:synthetic} illustrates the intimate connections between
logic, computability and categories via \emph{synthetic computability
  theory}~\cite{Bauer2006}. Specifically, we give a synthetic proof of a
fundamental result in computability theory: a subset of the natural numbers is
computable if and only if it and its complement are computably enumerable.


\section{Categorical logic in a nutshell}\label{sec:categorical-logic}
Suppose we have a formula \(\phi(x)\) with a free variable \(x\). If we
interpret \(x\) to range over some set \(X\), then \(\phi(x)\) determines a
subset of \(X\), namely \(\set{x \in X \mid \phi(x)}\), the subset of elements
\(x\) of \(X\) for which \(\phi(x)\) holds.
%
A conjunction of formulas also determines a subset: the
intersection of the subsets determined by \(\phi\) and \(\psi\):
\[
  \set{x \in X \mid \phi(x) \land \psi(x)}
  = \set{x \in X \mid \phi(x)} \cap \set{x \in X \mid \psi(x)}.
\]
Similarly, with conjunction and union of course.
%
The formulas \(\top\) (truth) and \(\bot\) (falsity) determine the two extreme
subsets \(X\) and \(\emptyset\), respectively.
%
The subsets of \(X\) are a partial order when equipped with the subset
relation. Its greatest element is \(X\), its least element is \(\emptyset\), the
intersection of two subsets is the greatest lower bound of the two subsets,
while the least upper bound is given by the union.
%
Thus, we see that we can interpret a formula as a subset and that the logical
connectives are interpreted using operations (characterized by universal
properties) on subsets.

\paragraph{From subsets to monos}
In categorical logic, we generalize this from the category of sets to arbitrary,
sufficiently rich categories.
%
Instead of subsets we consider monomorphisms, or really
\emph{subobjects}\footnote{A subobject of \(X\) is a monomorphism into
  \(X\) up to isomorphism in the slice category over \(X\).}.
%
The monos into a fixed object \(X\) in a category form a preorder by setting
\((M \hookrightarrow X) \preceq (N \hookrightarrow X)\) if we have a map
\(M \to N\) making the triangle
\[
  \begin{tikzcd}[row sep=4mm,column sep=4mm]
    M \ar[dr] \ar[rr,hookrightarrow] & & X \\
    & N \ar[ur,hookrightarrow]
  \end{tikzcd}
\]
commute. (One can check that the map \(M \to N\) is necessarily a mono.)

This preorder generalizes the subsets with their subset relation.
%
The relation \(\preceq\) is reflexive and transitive, but not necessarily
antisymmetric. But we can consider the \emph{poset reflection} of this preorder,
where we quotient such that monos \(M \hookrightarrow X\) and
\(N \hookrightarrow X\) are identified when \(M \preceq N\) and \(N \preceq M\)
both hold.
%
One may check that an element of the resulting poset is precisely a subobject.

\paragraph{Logical connectives and monos}
The formula \(\top\) is interpreted as the greatest element in the preorder of
monos into \(X\), i.e.\ as \(\id_X \colon X \to X\).
%
The formula \(\bot\) is interpreted as the least element in the preorder of
monos into \(X\), which in a cartesian closed category with an initial object
\(0\) (like the category of assemblies) indeed exists and is given by the unique
map\footnote{The proof that this map is indeed a mono is short, but surprisingly
  tricky. The interested reader might like to prove it for themselves. In case
  they get stuck, see \cite[Theorem~6.3]{McLarty1992} or \cite{NCF}.}
\(0 \to X\).

Following the universal properties of subsets, we deduce that conjunction should
be interpreted using the greatest lower bound in the preorder of monos.
%
Assuming our category has pullbacks, one can check that taking the pullback
\[
  \begin{tikzcd}
    M \land N \pbcorner \ar[d,hookrightarrow] \ar[r,hookrightarrow]
    & N \ar[d,hookrightarrow] \\
    M \ar[r,hookrightarrow] & X
  \end{tikzcd}
\]
of two monos into \(X\) gives their greatest lower bound
\(M \land N \hookrightarrow X\).
%
(In the diagram all maps are monos because those are stable under pullback.)

Interpreting disjunction is slightly more involved: for two monos
\({M,N \hookrightarrow X}\) the induced map \(M + N \to X\) from the coproduct is
not in general a mono. Instead, we require that the category factors any map as
a regular epi followed by a mono (like we have in the category of assemblies by
\cref{exer:reg-epi-mono-factorization}) and we factor \(M + N \to X\) as
\(M + N \twoheadrightarrow M \lor N \hookrightarrow X\).

It is an insight of Lawvere~\cite{Lawvere1969} that the quantifiers \(\forall\)
and \(\exists\) can also be suitably captured by universal properties using
adjunctions.
%
Any morphism of the category \(f \colon X \to Y\) induces a monotone map
\(f^\ast \colon \Mono(Y) \to \Mono(X)\) on the preorders of monos into \(Y\) and
\(X\) respectively by pulling back along \(f\).
%
Viewing these preorders as categories, the map \(f^\ast\) has a right adjoint
\(\exists_f\) by defining \(\exists_f(m)\) to be the monomorphism part of the
factorization of \(f \circ m\) as a regular epi followed by a mono.
%
For the universal quantifier, we then require \(f^\ast\) to also have a right
adjoint which we denote by \(\forall_f\).
%
For locally cartesian closed categories, like the category of assemblies
(\cref{base-change-adjoints}), we already have a right adjoint \(\prod_f\) to the
pullback functor which, being a right adjoint, preserves monos and hence
restricts to a monotone map on the preorders of monos.

\paragraph{Sorts, relations and terms}
We assume to be given a many-sorted language. This means that variables,
function symbols and relation symbols are typed by an assignment of
\emph{sorts}.
%
The sort of a variable indicates what it is supposed to range over.
%
Function symbols have a source and target sort and relation symbols have as many
sorts as their arity.
%
For example, if we are interested in arithmetic we might have a single sort
\(N\) for the natural numbers and a function symbol \(s\) for the successor map
with source and target sort \(N\).
%
We can build terms by e.g.\ applying a function symbol to a variable, subject to
the condition that the sorts match up of course.

An interpretation of the language in the category is given by several assignments:
\begin{itemize}
\item each sort \(X\) is interpreted as an object \(\sem{X}\) of the category;
\item each function symbol \(f\) with source sort \(S\) and target sort \(T\) is
  interpreted as an arrow \(\sem{f} \colon \sem{S} \to \sem{T}\) in the
  category;
\item each relation symbol \(R\) with sorts \(X_1,\dots,X_n\) is interpreted as
  a subobject \(\sem{R}\) of the product
  \(\sem{X_1}\times\cdots\times\sem{X_n}\).
\end{itemize}

We can also interpret equality of two terms \(s\) and \(t\) by taking the
equalizer of their interpretations \(\sem{s}\) and \(\sem{t}\).

For the example of the language of arithmetic, it would make good sense to
require the category to have a natural numbers object and to use this as the
interpretation of the sort of the natural numbers.
%
If we added sorts \(N^N\), \(N^{\pa*{N^N}}\), etc.\ for functions then we would
interpret these using exponentials in a cartesian closed category.

The purpose of the above is that it tells us that a formula \(\phi(x)\) with a
free variable \(x\) of sort \(X\) should be interpreted as a monomorphism into
\(\sem{X}\), i.e.\ it tells us which object to consider.
%
The actual interpretation of \(\phi(x)\) as mono into \(\sem{X}\) can then be
calculated by recursion on the structure of \(\phi\).
%
(If \(\phi\) has more than one free variable, say \(x_1,\dots,x_n\) of sorts
\(X_1,\dots,X_n\), respectively, then we simply consider the object
\(\sem{X_1} \times \cdots \times \sem{X_n}\).)

We write \(\forall(x : X).\phi(x)\) and \(\exists(x : X).\phi(x)\) when
quantifying over a sort \(X\). The colon (:) should remind us that
these quantifications are interpreted in the category.

\paragraph{Internal language}
Given a category \(\mathcal C\), a particular language we might consider is the
\emph{internal language} of \(\mathcal C\) where we add a sort for each object
of \(\mathcal C\), a function symbol for each morphism of \(\mathcal C\),
and finally a relation symbol for each subobject of \(\mathcal C\).
%
This language has a natural interpretation in \(\mathcal C\): if we added a sort
for an object \(X\), then the interpretation of that sort is \(X\); and
similarly for the function and relation symbols.
%
In this case we do not distinguish notationally between the object and the sort,
the function symbol and the morphism, and the relation symbol and the subobject.

\paragraph{Soundness}
One can show that the poset of subobjects is always a \emph{Heyting algebra}
whenever the category has sufficient structure (e.g.\ when it is regular and the
adjoint \(\forall_f\) exists).
%
As a consequence we have:
\begin{theorem*}[Soundness theorem]
  If we can prove \(\psi\) from \(\phi\) in constructive logic, then
  \(\sem{\phi} \preceq \sem{\psi}\) holds in the poset of subobjects.
\end{theorem*}

The subobjects do \emph{not} usually form a \emph{Boolean algebra}, however,
which means we do not have soundness with respect to classical logic, i.e.\ the
law of excluded middle may not be validated by the interpretation in the
category.


\section{Categorical logic in categories of assemblies}\label{sec:cat-logic-in-asm}

We wish to describe the logical side of the category of assemblies. By
\cref{sec:categorical-logic} this means studying the monomorphisms of
\(\Asm{\AA}\).
%
As often, it is advisable to first find a more convenient description of the
monos in this category. For example, in the category of sets, it is often useful
to work with subsets instead of injections, and in the category of presheaves on
a category we similarly often choose to work with subpresheaves.
%
Following Bauer~\cite{Bauer2023}, we choose the preorder of \emph{realizability
  predicates} on an assembly \(X\) as a convenient substitute for the preorder of
monos into \(X\).
%


\begin{definition}[Realizability predicate]
  A \textbf{realizability predicate} on an assembly~\(X\) is a function
  \(\carrier{X} \to \PA\).
\end{definition}

Note that the definition of a realizability predicate makes sense even when
\(X\) is just a set and not an assembly. However, we will shortly define a
preorder on realizability predicates on an assembly \(X\) which \emph{does} make
essential use of the realizers of \(X\).

\begin{notation}
  We will typically write \(\phi\) and \(\psi\) for realizability predicates.
\end{notation}

We think of a realizability predicate \(\phi\) on \(X\) as a logical predicate
on \(X\), and of \(\phi(x)\) as the set of witnesses that \(\phi\) holds for the
element \(x\).

\begin{definition}[Preorder on realizability predicates, \(\phi \predleq \psi\)]
  For two realizability predicates \(\phi\) and \(\psi\) on an assembly \(X\),
  we put \( \phi \predleq \psi \) exactly if there exists \(\pca{r} \in \AA\)
  such that, for every \(x \in \carrier{X}\), realizer \(\pca{a} \realizes_X x\)
  and witness \(\pca{b} \in \phi(x)\), we have
  \(\pca{r}\pca{a}\pca{b} \in \psi(x)\),
  %   \phi \predleq \psi \iff
  %   \exists(\pca{r} \in \pca{\AA}).
  %   \forall (x \in X).
  %   \forall(\pca{a} \realizes_X x).
  %   \forall(\pca{b} \in \phi(x)).
  %   \pca{r}\pca{a}\pca{b} \in \psi(x),
  % \]
  where we implicitly include the requirement that \(\pca{r}\pca{a}\pca{b}\) is defined.
\end{definition}

Intuitively, we have \(\phi \predleq \psi\) exactly if we can effectively
calculate a witness that \(\psi\) holds at \(x\) from a witness that \(\phi\)
holds at \(x\) \emph{and} a realizer of \(x\).
%
It is at this last point that we make essential use of the fact that \(X\) is an
assembly and not just a set.

%
It is straightforward (and similar to checking that assemblies with assembly
maps form a category) to check that \(\predleq\) is indeed a preorder, i.e.\
that it is reflexive and transitive.

\begin{notation}[Preorder of realizability predicates, \(\realpred(X)\)]
  We write \(\realpred(X)\) for the preorder of realizability predicates on an
  assembly \(X\).
\end{notation}

Every monomorphism \(m \colon Y \hookrightarrow X\) into an assembly \(X\)
determines a realizability predicate \(\phi_m\) on \(X\) by:
\begin{align*}
  \phi_m(x) &\coloneqq
  \set{\pca{a} \in \AA \mid y \in m^{-1}(x) \text{ and} \pca{a} \realizes_Y y} \\
  &\hspace{3.5pt}=
  \set{\pca{a} \in \AA \mid
    \exists(y \in \carrier{Y}).\pca{a} \realizes_Y y \text{ and } m(y) = x}.
\end{align*}
%
Notice that the \(y \in \carrier{Y}\) is necessarily unique (if it exists),
because \(m\) is injective.

Conversely, every realizability predicate on \(X\) determines a monomorphism of
assemblies \([\phi] \hookrightarrow X\) via:
\begin{align*}
  \carrier{[\phi]} &\hspace{8.5pt}\coloneqq\hspace{8.5pt}
                     \set{x \in \carrier{X} \mid \phi(x) \neq \emptyset} \\
  \pcapair\pca{a}\pca{b} \realizes_{[\phi]} x
  &\iff \pca{a} \realizes_X x
  \text{ and }
  \pca{b} \in \phi(x),
\end{align*}
where the inclusion \([\phi] \hookrightarrow X\) is tracked by \(\pcafst\).

\begin{proposition}\label{preorders-mono-predicates-iso}
  For an assembly \(X\), the above constructions constitute an equivalence
  between the preorder of monos into \(X\) and the preorder of
  realizability predicates on \(X\).
\end{proposition}
\begin{exercise}\label{exer:preorders-monos-predicates-iso}
  Prove the proposition.
\end{exercise}


\subsection{The Heyting prealgebra of realizability predicates}%
\label{sec:Heyting-prealgebra-realizability-predicates}

We describe the structure on the preorder of realizability
predicates required for interpreting first order logic.
%
The interpretation itself is detailed in
\cref{sec:realizability-interpretation}.

\begin{definition}
  There are two extreme examples of realizability predicates on an assembly \(X\):
  \begin{align*}
    \bot(x) &\coloneqq \emptyset, \\
    \top(x) &\coloneqq \AA.
  \end{align*}
  For realizability predicates \(\phi\) and \(\psi\) on a set \(X\), we define
  three new realizability predicates on \(X\) by:
  \begin{align*}
  %\item
    (\phi \land \psi)(x) &\coloneqq \set{\pcapair\pca{a}\pca{b} \mid
      \pca{a} \in \phi(x) \text{ and } \pca{b} \in \psi(x)}, \\
  %\item
    (\phi \lor \psi)(x) &\coloneqq
    \set{\pcaleft\pca{a} \mid \pca{a} \in \phi(x)} \cup
    \set{\pcaright\pca{b} \mid \pca{b} \in \psi(x)},
                          \quad(\text{recall~\cref{coproducts}}) \\
    % \item
    (\phi \Rightarrow \psi)(x) &\coloneqq
    \set{\pca{r} \in \AA \mid \text{for every}
    \pca{a} \in \phi(x) \text{ we have} \pca{r}\pca{a} \in \psi(x)}.
  \end{align*}
\end{definition}

\begin{exercise}\label{exer:predicates-heyting-prealgebra}
  Check that \(\bot\), \(\top\), \(\land\), \(\lor\) and \(\Rightarrow\) as
  defined above make the preorder \(\realpred(X)\) of realizability predicates
  on an assembly \(X\) into a Heyting prealgebra.

  (For a short definition of the latter: it is a preorder that, when viewed as a
  category, has finite (co)limits and exponentials.)
\end{exercise}

In light of \cref{preorders-mono-predicates-iso} and the fact that the Heyting
(pre)algebra operations are characterized by universal properties we know that
the operations defined above correspond to the relevant operations on monos. For
example, \([\bot]\) is the monomorphism \(0 \hookrightarrow X\) and
\([\phi\land\psi]\) is the meet (= greatest lower bound) of the monos \([\phi]\)
and \([\psi]\).

The interpretation of the quantifiers is given by adjoints, as briefly discussed
in \cref{sec:categorical-logic}. We explicitly describe these adjoints in terms
of the preorder of realizability predicates, where the
natural-bijection-between-hom-sets definition of adjoints simplifies to the
condition of a (monotone) \emph{Galois connection}: a map \(l\) between preorders is
left adjoint to \(r\) exactly when \(l(x) \leq y \iff x \leq r(y)\) holds.

\begin{proposition}
  For an assembly map \(f \colon X \to Y\) and a realizability predicate
  \(\phi\) on~\(X\), the realizability predicate \(\forall_f(\phi)\) on \(Y\)
  defined by
  \[
    {\forall_f(\phi)}\,(y) \coloneqq
    \set{\pca{t} \in \AA \mid
      \text{for every } x \in f^{-1}(y) \text{ and } \pca{a} \realizes_X x, \text{we have}
      \pca{t}\pca{a} \in \phi(x)}
  \]
  satisfies
  \[
    f^\ast(\psi) \predleq \phi \iff \psi \predleq \forall_f(\phi)
  \]
  for all realizability predicates \(\psi\) on \(Y\).

  In other words, \(\forall_f \colon \realpred(X) \to \realpred(Y)\) is a right
  adjoint to \(f^\ast \colon \realpred(Y) \to \realpred(X)\).
\end{proposition}
\begin{proof}
  We spell out what each of \(f^\ast(\psi) \predleq \phi\) and
  \(\psi \predleq \forall_f(\phi)\) amounts to.

  The former requires the existence of an element \(\pca{r_1} \in \AA\) such that
  for every \(x \in \carrier{X}\), \(\pca{a} \realizes_X x\) and
  \(\pca{c} \in \psi(f(x))\), we have \(\pca{r_1}\pca{a}\pca{c} \in \phi(x)\).

  The latter requires the existence of an element \(\pca{r_2} \in \AA\) such
  that for every \(y \in \carrier{Y}\), \(\pca{b} \realizes_Y y\) and
  \(\pca{c} \in \psi(y)\) we have
  \(\pca{r_2}\pca{b}\pca{c} \in \forall_f(\phi)\,(y)\).
  %
  That is, \(\pca{r_2}\pca{b}\pca{c}\) should satisfy
  \(\pca{r_2}\pca{b}\pca{c}\pca{a} \in \phi(x)\) for all
  \(\pca{a} \realizes_X x\) with \(f(x) = y\).

  Now notice that given such an \(\pca{r_1}\), the program
  \(\lambdapca{vwu}{\pca{r_1}u\,w}\) does the job of \(\pca{r_2}\).

  Conversely, given such an \(\pca{r_2}\), the program
  \(\lambdapca{uw}{\pca{r_2}(\pca{t_f}u)\,w\,u}\), where \(\pca{t_f}\) is a
  tracker of \(f\), does the job of \(\pca{r_1}\).
\end{proof}

For an alternative proof, one may verify that \(\forall_f(\phi)\) is the
realizability predicate determined by the monomorphism \(\prod_f([\phi])\),
where we recall \(\prod_f\) from \cref{base-change-adjoints}.

\begin{proposition}
  For an assembly map \(f \colon X \to Y\) and a realizability predicate
  \(\phi\) on~\(X\), the realizability predicate \(\exists_f(\phi)\) on \(Y\)
  defined by
  \[
    {\exists_f(\phi)}\,(y) \coloneqq
    \bigcup_{x \in f^{-1}(y)}\set{\pcapair\pca{a}\pca{b}
      \mid \pca{a} \realizes_X x \text{ and } \pca{b} \in \phi(x)}
  \]
  satisfies
  \[
    \exists_f(\phi) \predleq \psi \iff \phi \predleq f^\ast(\psi)
  \]
  for all realizability predicates \(\psi\) on \(Y\).

  In other words, \(\exists_f \colon \realpred(X) \to \realpred(Y)\) is a left
  adjoint to \(f^\ast \colon \realpred(Y) \to \realpred(X)\).
\end{proposition}
\begin{exercise}\label{exer:exists-predicate}
  Prove the proposition either directly or by checking that \(\exists_f(\phi)\)
  is the realizability predicate \(\phi_m\) determined by \(m\) in the
  factorization of the top composite
  \[
    \begin{tikzcd}[row sep=2mm,column sep=8mm]
      {[\phi]} \ar[dr] \ar[r,hookrightarrow] & X \ar[r,"f"] & Y \\
      & M \ar[ur,hookrightarrow,"m"']
    \end{tikzcd}
  \]
  as a regular epimorphism followed by a monomorphism
  (recall~\cref{exer:reg-epi-mono-factorization}).
  %
  % In other words, that \(\exists_f(\phi)\) is the realizability predicate
  % corresponding to the mono \(\exists_f([\phi])\).
\end{exercise}

In the situations that will be of us interest to us, \(f\) will also be
surjective when we consider \(\forall_f\) and \(\exists_f\).
%
If this is the case, then \(\forall_f(\phi)\) admits a definition that is more
symmetric to that of \(\exists_f(\phi)\), because in this case we have
%\begin{exercise}\label{exer:forall-predicate-surjective}
\[
  \forall_f(\phi)\,(y) = \bigcap_{x \in f^{-1}(y)}
  \set{\pca{t} \in \AA \mid
    \pca{t}\pca{a} \in \phi(x) \text{ for all} \pca{a} \realizes_X x}
\]
for any realizability predicate \(\phi\) on \(X\).
% \end{exercise}

\subsection{The realizability interpretation of logic}\label{sec:realizability-interpretation}

Suppose we are given a formula \(\phi\) in a language whose sorts and relation
and function symbols have been assigned an interpretation in \(\Asm{\AA}\).
%
If \(\phi\) has free variables \(x_1,\dots,x_n\) of sorts \(X_1,\dots,X_n\),
then, following the preceding development, we may interpret \(\phi\) as a
realizability predicate on the assembly \(\sem{X_1} \times \cdots\times \sem{X_n}\).
%
We write \(\sem{\phi}\) for this predicate. Thus, for each
\(\vec x \in \carrier*{\sem{X_1}\times\cdots\times\sem{X_n}}\), we have a subset
\(\sem{\phi}(\vec x) \subseteq \AA\) of realizers.

Using the constructions of
\cref{sec:Heyting-prealgebra-realizability-predicates} we prove the following
recursive characterization of membership of such subsets, where we use
\(\vec x_{|_\phi}\) for the restriction of a tuple to those elements that
pertain to the domain of \(\sem{\phi}\) (and similarly for terms).

\begin{proposition}\label{realizability-logic}
  The realizability predicates arising from first order logic obey
  \begin{alignat*}{2}
    &\pca{r} \in \sem{s = t}(\vec x)
    &&\quad\iff\quad \sem{t}\pa*{\vec x_{|_t}} = \sem{s}\pa*{\vec x_{|_s}} \\[5pt]
    &\pca{r} \in \sem{\bot}(\vec x)
    &&\quad\iff\quad \text{never}, \\[5pt]
    &\pca{r} \in \sem{\top}(\vec x)
    &&\quad\iff\quad \text{always}, \\[5pt]
    &\pca{r} \in \sem{\phi \land \psi}(\vec x)
    &&\quad\iff\quad \pcafst\pca{r}\in\sem{\phi}\pa[\big]{\vec x_{|_\phi}} \text{ and }
       \pcasnd\pca{r}\in\sem{\psi}\pa[\big]{\vec x_{|_\psi}} \\[5pt]
    &\pca{r} \in \sem{\phi \lor \psi}(\vec x)
    &&\quad\iff\quad
       \pa[\Big]{\pca{r} = \pcaleft\pca{r'}
       \text{ and } \pca{r'}\in \sem{\phi}\pa[\big]{\vec x_{|_\phi}}}
       \text{ or } \\%[5pt]
    & &&\hspace{60pt} \pa[\Big]{\pca{r} = \pcaright\pca{r'}
         \text{ and } \pca{r'}\in \sem{\psi}\pa[\big]{\vec x_{|_\psi}}}, \\[5pt]
    &\pca{r} \in \sem{\phi \Rightarrow \psi}(\vec x)
    &&\quad\iff\quad \text{if } \pca{a} \in \sem{\phi}\pa[\big]{\vec x_{|_\phi}}, \text{then }
       \pca{r}\pca{a} \in \sem{\psi}\pa[\big]{\vec x_{|_\psi}}, \\[5pt]
    &\pca{r} \in \sem{\forall(x : X).\phi}(\vec x)
    &&\quad\iff\quad \text{if } x \in \carrier{\sem{X}}
       \text{ and } \pca{a} \realizes_{\sem{X}} x, \text{then }
       \pca{r}\pca{a} \in \sem{\phi}(\vec x,x), \\[5pt]
    &\pca{r} \in \sem{\exists(x : X).\phi}(\vec x)
    &&\quad\iff\quad \text{there is an } x \in \carrier{\sem{X}}
       \text{ such that} \\%[5pt]
    & &&\phantom{\hspace{60pt}} \pcafst\pca{r} \realizes_{\sem{X}} x \text{ and }
         \pcasnd\pca{r} \in \sem{\phi}(\vec x,x).
  \end{alignat*}
\end{proposition}

A \emph{closed} formula \(\phi\) corresponds to a realizability predicate on
\(\One\) and may thus be identified with a single subset of \(\AA\).
%
We say that such a \(\phi\) is \textbf{realized}, or \textbf{valid in
  \(\Asm{\AA}\)} if we have an element of this subset.

If we take \(\AA \coloneqq \Kone\) and consider the language of arithmetic, then
we recover Kleene's original \emph{number realizability}~\cite{Kleene1945}.
%
The choice of \(\AA \coloneqq \Ktwo\) and the language of analysis recovers
Kleene's \emph{function realizability}~\cite{KleeneVesley1965}.
%

\cref{realizability-logic} may appear to be a formalization of the so-called
\emph{Brouwer--Heyting--Kolmogorov (BHK) interpretation} and is often presented
as such, although van Oosten argues this is not historically
accurate~\cite[p.~241]{vanOosten2002}. We also worth pointing out that Kleene's
realizability predates the Curry--Howard correspondence.


It is worth spelling out the realizability interpretation of (double) negation.
%
Recall that \(\lnot\phi\) is defined as \(\phi \Rightarrow \bot\), so that we
have:
\begin{lemma}%[Realizability interpretation of (double) negation]
  The realizability predicates interpreting (double) negations satisfy
  \begin{alignat*}{2}
    &\pca{r} \in \sem{\lnot\phi}(\vec x) &&\quad\iff\quad
    \sem{\phi}(\vec x) = \emptyset, \\
    &\pca{r} \in \sem{\lnot\lnot\phi}(\vec x) &&\quad\iff\quad
    \sem{\phi}(\vec x) \neq \emptyset.
  \end{alignat*}
\end{lemma}

In particular, we see that \(\sem{\lnot\lnot\phi}(\vec x)\) has no computational
content as any element of \(\AA\) acts as a realizer whenever
\(\sem{\phi}(\vec x)\) is nonempty.
%
This suggests a connection to the functor \(\nabla \colon \Set \to \Asm{\AA}\)
(from \cref{sec:relation-to-Set}) which we indeed explore in
\cref{sec:double-negation-stable}.

We repeat that the realizability interpretation of first order logic---by virtue
of the Heyting prealgebra structure---validates constructive logic, i.e.\ first
order logic without excluded middle.
%
In fact, as \cref{exer:not-double-negation-stable-predicate} shows, the logic
governing realizability predicates is never classical unless the pca is
trivial (in which case the category of assemblies is the familiar category of
sets).

\subsection{Revisiting (regular) epis and monos}\label{sec:revisiting-epis-monos}

A nice way of getting some familiarity with realizability logic is by proving
the following characterizations of (regular) epis and monos in the internal
logic of \(\Asm{\AA}\).
%
We recall our convention to use the same letters for both the formal symbol in
the internal language and its interpretation in the category, e.g.\ if \(f\) is
an assembly map, then we formally have a function symbol
\(\ulcorner f \urcorner\) in our language with interpretation
\(\sem{\ulcorner f \urcorner} \coloneqq f\); but we'll simply reuse the letter
\(f\) for this function symbol.

To appreciate \cref{exer:epis-monos-logically}, recall that an assembly map is
an epi if and only if it's surjective, while it's regular epi if and only if
it's surjective \emph{and} we have an element in~\(\AA\) that witnesses
surjectivity (\cref{exer:characterize-regular-epis}).
%
This difference in computational content is reflected in the (non)use of the
double negation in the first items of the exercise below.

\begin{exercise}\label{exer:epis-monos-logically}
  Use \cref{exer:characterize-regular-monos,exer:characterize-regular-epis} to
  show that for an assembly map \({f \colon X \to Y}\), we have the following
  logical characterizations:
  \begin{enumerate}[(i)]
  \item \(f\) is a regular epimorphism if and only if
    \[
      \forall(y : Y).\exists(x : X).f(x) = y
    \]
    is realized;
  \item \(f\) is an epimorphism if and only if
    \[
      \forall(y : Y).\lnot\lnot({\exists(x : X).f(x) = y})
    \]
    is realized;
  \item \(f\) is a monomorphism if and only if
    \[
      \forall(x, x' : X).(f(x) = f(x') \Rightarrow x = x')
    \]
    is realized;
  \item \(f\) is a regular monomorphism if and only if
    \[
      \forall(y : Y).\pa*{\lnot\lnot\pa*{\exists(x : X).f(x) = y} \Rightarrow \exists!(x : X).f(x) = y}
    \]
    is realized.

    The quantifier \(\exists!\) means ``there exists a unique \dots with \dots''.

    Phrased in English, \(f\) is a regular monomorphism if and only if the statement
    \begin{quote}{``For all \(y\), if the preimage of \(f\) at \(y\) is nonempty,
        then we can (effectively) find a unique \(x\) with \(f(x) = y\).''}
    \end{quote}
    is realized.
  \end{enumerate}
\end{exercise}


\section{Two-element assemblies as classifiers}\label{sec:two-element-assemblies}

This section introduces three sets of realizability predicates, namely the
\emph{\(\lnot\lnot\)-stable}, \emph{decidable} and \emph{semidecidable}
predicates.
%
These are shown to be \emph{classified} by three different assemblies all of
which have the set \(\set{0,1}\) as their carriers, but different realizers.
%
For example, the \(\lnot\lnot\)-stable realizability predicates are classified
by \(\nabla\set{0,1}\), while the assembly \(\Two\) of booleans classifies the
decidable realizability predicates.
%
We moreover give explicit connections to computability theory by specializing to
the category of assemblies over Kleene's first model.

\subsection{Double negation stable realizability predicates}\label{sec:double-negation-stable}

We have already seen that the monomorphisms of assemblies are given exactly by
realizability predicates. The \emph{regular} monos can be characterized as a
subset of those realizability predicates, namely those that are
\(\lnot\lnot\)-stable.

\begin{definition}[\(\lnot\lnot\)-stability]
  A realizability predicate \(\phi\) on an assembly \(X\) is said to be
  \textbf{\(\lnot\lnot\)-stable} if
  \[
    \forall(x : X).(\lnot\lnot\phi(x) \Rightarrow \phi(x))
  \]
  is realized.
\end{definition}

In some of the literature (on topos theory), one also sees the name
\emph{\(\lnot\lnot\)-closed}.
%
Bauer uses the word \emph{classical} in~\cite{Bauer2023}.
%
This is reasonable terminology, because in classical logic everything is
\(\lnot\lnot\)-stable. Moreover, as we will see the \(\lnot\lnot\)-stable
realizability predicates correspond to ordinary `classical' subsets.

In realizability, not all predicates are \(\lnot\lnot\)-stable, as we ask you to
verify by means of the following exercise:

\begin{exercise}\label{exer:not-double-negation-stable-predicate}
  Show that if all realizability predicates of \(\Asm{\AA}\) are
  \(\lnot\lnot\)-stable, then the pca \(\AA\) is trivial.

  \emph{Hint}: For \(\pca{a},\pca{b} \in \AA\), consider a suitable
  realizability predicate on \(\nabla\set{0,1}\).
\end{exercise}

In fact, we have already seen the \(\lnot\lnot\)-stable realizability predicates
because (seen as monomorphisms) they are precisely the regular monos, as we ask you
to verify.

\begin{exercise}\label{exer:double-negation-stable-iff-regular-mono}
  Prove that a realizability predicate \(\phi\) on an assembly \(X\) is
  \(\lnot\lnot\)-stable if and only if its corresponding monomorphism
  \([\phi] \hookrightarrow X\) is regular.
\end{exercise}

The \(\lnot\lnot\)-stable predicates have no computational content as made
precise by the following result:

\begin{proposition}\label{not-not-stable-ordinary-subsets}
  Every \(\lnot\lnot\)-stable realizability predicate \(\phi\) on an
  assembly \(X\) is uniquely determined by a subset \(A \subseteq \carrier{X}\)
  such that we have a pullback diagram
  \[
    \begin{tikzcd}
      {[\phi]} \ar[r] \ar[d,hookrightarrow] \pbcorner
      & \nabla A \ar[d,hookrightarrow] \\
      X \ar[r,"\eta_X"] & \nabla\carrier{X}
    \end{tikzcd}
  \]
  in \(\Asm{\AA}\).% , where we recall that \([\phi] \hookrightarrow X\) denotes
  % the mono corresponding to \(\phi\).
\end{proposition}
\begin{proof}
  Given a \(\lnot\lnot\)-stable realizability predicate \(\phi\) on \(X\), we
  define
  \[
    A \coloneqq \set{x \in \carrier{X} \mid \phi(x) \neq \emptyset} = \carrier{[\phi]}.
  \]
  %
  We may compute the pullback of \(\nabla A \hookrightarrow \nabla\carrier{X}\)
  along \(\eta_X\) as the assembly \(P\) with
  \[
    \carrier{P} \coloneqq A \quad\text{and}\quad
    \pca{a} \realizes_P x \iff a \realizes_X x.
  \]
  The identity on \(A\) gives functions between \(\carrier{[\phi]}\) and
  \(\carrier{P}\). It remains to see that they are tracked.
  %
  Towards \(P\), the map is tracked by \(\pcafst\). In the other direction, we
  get a tracker by the assumption that \(\phi\) is \(\lnot\lnot\)-stable.

  For the converse, we note that the above computation indeed shows that such a
  pullback corresponds to a \(\lnot\lnot\)-stable realizability predicate,
  because the realizers of the pullback are just the realizers of \(X\).
\end{proof}


In fact, the \(\lnot\lnot\)-stable realizability predicates arise as pullbacks
of a single map:

\begin{proposition}\label{nabla-two-classifies-not-not-stable}
  Every \(\lnot\lnot\)-stable realizability predicate \(\phi\) on an assembly
  \(X\) is uniquely determined by a map \(\chi \colon X \to \nabla\set{0,1}\)
  such that we have a pullback diagram
  \[
    \begin{tikzcd}
      {[\phi]} \ar[r] \ar[d,hookrightarrow] \pbcorner
      & \One \ar[d,hookrightarrow,"{\singleton \,\mapsto\, 1}"] \\
      X \ar[r,"\chi"] & \nabla\set{0,1}
    \end{tikzcd}
  \]
  in \(\Asm{\AA}\).
\end{proposition}
\begin{proof}
  Given \(A \subseteq \carrier{X}\), the map
  \(
    \chi' \colon \nabla X \to \nabla\set{0,1}\) with \(\chi'(x)
    = \begin{cases}
      0 &\text{if } x \not\in A \\
      1 &\text{if } x \in A
    \end{cases}
    \)

  gives a pullback square
  \[
    \begin{tikzcd}
      \nabla A \ar[r] \ar[d,hookrightarrow] \pbcorner
      & \One \ar[d,hookrightarrow,"{\singleton \,\mapsto\, 1}"] \\
      \nabla\carrier{X} \ar[r,"\chi'"] & \nabla\set{0,1}
    \end{tikzcd}
  \]
  so the result follows from
  \cref{not-not-stable-ordinary-subsets} and pullback pasting.
\end{proof}

The map \(\One \to \nabla\set{0,1}\) is said to be a \textbf{classifier} for the
\(\lnot\lnot\)-stable realizability predicates (subobjects).

Finally, we can also internalize \cref{not-not-stable-ordinary-subsets} as:
\begin{exercise}\label{exer:nabla-two-nat-exp}
  For an assembly \(X\), the exponential \(\pa*{\nabla\set{0,1}}^X\) is
  isomorphic to \(\nabla(\powerset(\carrier{X}))\).
\end{exercise}

\subsection{Decidable realizability predicates}\label{sec:decidable-realizability-predicates}

After the \(\lnot\lnot\)-stable realizability predicates we now consider the
subset of \emph{decidable} realizability predicates. We show these to be
classified by the assembly of booleans and explore the connection to computable
subsets in assemblies over Kleene's first model.

\begin{definition}[Decidability]
  A realizability predicate \(\phi\) on an assembly \(X\) is said to be
  \textbf{decidable} if
  \[
    \forall(x : X).(\phi(x) \lor \lnot\phi(x))
  \]
  is realized.
\end{definition}

The decidable realizability predicates form a subset of the \(\lnot\lnot\)-stable ones.

\begin{lemma}\label{decidable-implies-not-not-stable}
  Every decidable realizability predicate is \(\lnot\lnot\)-stable.
\end{lemma}
\begin{proof}
  This is a nice opportunity to make use of the soundness of constructive logic,
  so that we don't need to concern ourselves with realizers.
  %
  That is, we give a constructive proof that decidability implies
  \(\lnot\lnot\)-stability.
  %
  If \(\phi\) is decidable, then we only have to consider two cases: if
  \(\phi(x)\) holds, then we trivially get
  \(\lnot\lnot\phi(x) \Rightarrow \phi(x)\); while if \(\lnot\phi(x)\) hold,
  then the assumption \(\lnot\lnot\phi(x)\) leads to a contradiction, allowing
  us to conclude \(\phi(x)\).
\end{proof}

Notice that \cref{decidable-implies-not-not-stable} in combination with
\cref{exer:not-double-negation-stable-predicate} implies that not all
realizability predicates are decidable (unless the pca is trivial).

\begin{proposition}
  The decidable realizability predicates are classified by \(\Two\).
\end{proposition}
\begin{proof}
  Suppose \(\phi\) is a decidable realizability predicate on an assembly \(X\).
  %
  Then the function \(\chi \colon \carrier{X} \to \carrier{\Two}\) defined as

  \[
    \chi(x) \coloneqq
    \begin{cases}
      0 &\text{if } \phi(x) = \emptyset, \\
          1 &\text{if } \phi(x) \neq \emptyset;
    \end{cases}
  \]
  is tracked because \(\phi\) is decidable, so we get an assembly map
  \(\chi \colon X \to \Two\).
  %
  Computing the pullback \(P\) in
  \[
    \begin{tikzcd}
      P \ar[r] \ar[d,hookrightarrow] \pbcorner
      & \One \ar[d,hookrightarrow,"{\singleton \,\mapsto\, 1}"] \\
      X \ar[r,"\chi'"] & \Two
    \end{tikzcd}
  \]
  we get
  \[
    \carrier{P} \coloneqq \set{x \in \carrier{X} \mid \phi(x) \neq \emptyset}
    = \carrier{[\phi]}
    \quad\text{and}\quad
    \pca{a} \realizes_P x \iff \pca{a} \realizes_X x.
  \]
  But, recalling the proof of \cref{not-not-stable-ordinary-subsets}, we see
  that \(P\) and \([\phi]\) are isomorphic as \(\phi\) is \(\lnot\lnot\)-stable
  by~\cref{decidable-implies-not-not-stable}.

  Conversely, given such a pullback \([\phi]\), the tracker of \(X \to \Two\)
  witnesses the fact that \(\forall(x:X).\lnot\lnot\phi(x) \lor \lnot\phi(x)\)
  is realized.
  %
  But the square
  \[
    \begin{tikzcd}
      \One \pbcorner \ar[r] \ar[d] & \One \ar[d,"\star \mapsto 1"] \\
      \Two \ar[r,hookrightarrow] & \nabla\set{0,1}
    \end{tikzcd}
  \]
  is a pullback, so by pullback pasting we see that \([\phi]\) is classified by
  \(\nabla\set{0,1}\). Therefore, by~\cref{nabla-two-classifies-not-not-stable},
  the predicate \(\phi\) is \(\lnot\lnot\)-stable, so in fact
  \(\forall(x:X).\phi(x) \lor \lnot\phi(x)\) is realized, as desired.
\end{proof}

For the remainder of this subsection we take \(\AA \coloneqq \Kone\), i.e., we
work with assemblies over Kleene's first model (\cref{ex:Kleene-1}).

\begin{exercise}\label{exer:nno-in-Asm-K1}
  Show that the natural numbers object in \(\Asm{\Kone}\) is isomorphic to the
  assembly \(\NatAsm\) with carrier \(\Nat\) and realizers
  \(n \realizes_\NatAsm n\) for each \(n \in \Nat\).
\end{exercise}

Similarly, one may show that in \(\Asm{\Kone}\) we can take the numbers \(0\)
and \(1\) as the respective realizers of the elements \(0,1 \in \carrier{\Two}\)
of the assembly of booleans.

Recall from computability theory that a subset \(A \subseteq \Nat\) is
\textbf{computable} if we have a total (Turing) computable function
\(\chi \colon \Nat \to \Nat\) such that
%\(\chi(n) \in \set{0,1}\) for all
%\(n \in \Nat\) %(we include this requirement for convenience only)
%and
\(\chi(n) = 1 \iff n \in A\).
%
We say that \(\chi\) \textbf{computes} \(A\).
%
Note: instead of ``computable'', some (older) textbooks will use the terminology
``recursive'', or (potentially confusing for us) ``decidable''.

The following exercises show that the decidable realizability predicates of
\(\Asm{\Kone}\) correspond precisely to computable subsets.


\begin{exercise}\label{exer:decidable-is-computable}\leavevmode
  \begin{enumerate}[(i)]
  \item Show that the exponential \(\Two^\NatAsm\) is isomorphic to the assembly
    \(\CC\) with
    \begin{align*}
      \carrier{\CC} \,\coloneqq\, \set{&A \subseteq \Nat \mid A \text{ is computable}}, \text{and} \\
      n \realizes_{\CC} A \iff &\prenum{n} \text{ computes } A.
    \end{align*}
  \item Show that there is a bijection between computable subsets and pullback
    squares
    \[
      \begin{tikzcd}
        \bullet \pbcorner \ar[r] \ar[d,hookrightarrow]
        & \One \ar[d,"\star \mapsto 1"] \\
        \NatAsm \ar[r] & \Two
      \end{tikzcd}
    \]
  \item Conclude that there is a bijection between computable subsets and
    decidable realizability predicates on \(\NatAsm\).
  \end{enumerate}
\end{exercise}

% \begin{exercise}\label{exer:Two-computable-subsets}
%   Show that for an assembly \(X\), we have a bijection between
%   \begin{enumerate}[(i)]
%   \item pullbacks
%     \[
%       \begin{tikzcd}
%         \bullet \pbcorner \ar[r] \ar[d,hookrightarrow]
%         & \One \ar[d,"\star \mapsto 1"] \\
%         X \ar[r] & \Two
%       \end{tikzcd}
%     \]
%   \item subsets \(X' \subseteq X\) and computable subsets
%     \(A \subseteq \Nat\) such that for all \(x \in X\), we have
%     \begin{align*}
%       x \in X' &\Rightarrow \set{n \in \Nat \mid n \realizes_X x} \subseteq A, \text{and} \\
%       x \not\in X' &\Rightarrow \set{n \in \Nat \mid n \realizes_X x} \cap A = \emptyset.
%     \end{align*}
%   \end{enumerate}
%   In particular, for \(X \coloneqq \NatAsm\), we recover (cf.\
%   \cref{exer:Two-to-N-is-C}) the fact that assembly maps
%   \(\NatAsm \to \Two\) correspond to computable subsets.
% \end{exercise}

\subsection{Semidecidable realizability predicates}\label{sec:semidecidable-realizability-predicates}

We introduce a third and final class of realizability predicates: the
semidecidable ones.
\begin{definition}[Semidecidability]
  A realizability predicate \(\phi\) on an assembly \(X\) is said to be
  \textbf{semidecidable} if
  \[
    \forall(x : X).\exists(\sigma:\Two^\NatAsm).
    \pa*{\pa*{\exists(n : \NatAsm).\sigma(n) = 1}\iff\phi(x)}
  \]
  is realized.
\end{definition}

Notice the inherent essential asymmetry of semidecidability: the assertion that
the predicate is true can be made by making finitely many observations (keep
querying the sequence until we see a \(1\)); on the other hand, to conclude that
the predicate is false we would need evidence that the sequence is \(0\)
\emph{everywhere}.

\begin{definition}[Assembly of semidecidable truth values, \(\Sigma\)]
  The \textbf{assembly of semidecidable truth values}, denoted by \(\Sigma\),
  has carrier \(\set{0,1}\) and realizers
  \begin{align*}
    \pca{r} \realizes_\Sigma b \text{ such that}
    &\pca{r}\numeral{k} \in \set{\pcatrue,\pcafalse}
    \text{ for all \(k \in \Nat\), and we have} \\
    &\pa*{\exists(n \in \Nat).\pca{r}\numeral{n} = \pcatrue} \iff (b = 1).
  \end{align*}
\end{definition}

Thus, the realizers of \(\Sigma\) are codes for binary sequences of booleans and
such a code for a sequence realizes the element \(1\) precisely when the
sequence is true somewhere.
%
It should therefore come as no surprise that we have:
\begin{exercise}\label{exer:Sigma-classifies-semidecidable-realizability-predicates}
  The assembly \(\Sigma\) (with distinguished element
  \(1 \in \carrier{\Sigma}\)) classifies the semidecidable realizability
  predicates.
\end{exercise}

For the remainder of this subsection we again work with Kleene's first model only.
% and we explore the connections between
% \(\Sigma\) and computable enumerable subsets.

\begin{exercise}\label{exer:Sigma-in-Kleene-1}
  Show that \(\Sigma\) is isomorphic to the assembly \(\Sigma'\) with carrier
  \(\set{0,1}\) and realizers
  \[
    n \realizes_{\Sigma'} 0 \iff n \not\in K
    \quad\text{and}\quad
    n \realizes_{\Sigma'} 1 \iff n \in K,
  \]
  where \(K \coloneqq \set{n \in \Nat \mid \prenum{n}(n) \text{ is defined}}\)
  is the \emph{(diagonal) Halting set}.

  \emph{Note}: This requires a little bit of computability theory.
\end{exercise}

It follows from~\cref{exer:Sigma-in-Kleene-1} that while both inclusions
\[
  \Two \hookrightarrow \Sigma \hookrightarrow \nabla\set{0,1}
\]
are mono and epi, neither of them is regular mono or regular epi, as it would
imply (check!) that membership of the Halting set \(K\) is computable which it
(famously) isn't.

Recall from computability theory that a subset \(A \subseteq \Nat\) is
\textbf{computably enumerable} (or \textbf{c.e.} for short) if it is empty or if
we have a total computable function \(e \colon \Nat \to \Nat\) such that \(e\)
\textbf{enumerates} \(A\) in the sense that the image of \(e\) and \(A\) are
equal.
%
Note: instead of ``computably enumerable'', some (older) textbooks will use the terminology
``recursively enumerable'', or (potentially confusing for us) ``semidecidable''.
%
A standard example of a computably enumerable subset that is not computable is
the Halting set \(K\).

The following exercises explore the connections between the semidecidable
realizability predicates of \(\Asm{\Kone}\) and computably enumerable subsets.

\begin{exercise}\label{exer:Sigma-to-N-is-CE}
  Show that the exponential \(\Sigma^\NatAsm\) is isomorphic to the assembly
  \(\CE\) with
  \begin{align*}
    \carrier{\CE} \,\coloneqq\, \set{&A \subseteq \Nat \mid A \text{ is computably enumerable}}, \text{and} \\
    n \realizes_{\CE} A \iff &\prenum{n} \text{ enumerates } A.
  \end{align*}
\end{exercise}

\begin{exercise}\label{exer:Sigma-ce-subsets}
  Show that for an assembly \(X\), we have a bijection between
  \begin{enumerate}[(i)]
  \item pullbacks
    \[
      \begin{tikzcd}
        \bullet \pbcorner \ar[r] \ar[d,hookrightarrow]
        & \One \ar[d,"\star \mapsto 1"] \\
        X \ar[r] & \Sigma
      \end{tikzcd}
    \]
  \item subsets \(X' \subseteq X\) and c.e.\ subsets
    \(A \subseteq \Nat\) such that for all \(x \in X\), we have
    \begin{align*}
      x \in X' &\Rightarrow \set{n \in \Nat \mid n \realizes_X x} \subseteq A, \text{and} \\
      x \not\in X' &\Rightarrow \set{n \in \Nat \mid n \realizes_X x} \cap A = \emptyset.
    \end{align*}
  \end{enumerate}
  In particular, taking \(X \coloneqq \NatAsm\), we see that semidecidable
  realizability predicates on \(\NatAsm\) correspond to computably
  enumerable subsets.
\end{exercise}

Recall \textbf{Rice's Theorem} from computability theory:

%
\begin{theorem*}[Rice]
  Suppose that \(P\) is a subset of\/ \(\Nat\) such that
  \begin{enumerate}[(i)]
  \item \(P\) is an \emph{index set}: if \(n \in P\) and \(\prenum{n} = \prenum{m}\), then \(m \in P\);
  \item \(P\) is \emph{nontrivial}, i.e.\ \(P \neq \emptyset\) and \(P \neq \Nat\).
  \end{enumerate}
  Then \(P\) is not computable.
\end{theorem*}

Rice's Theorem is often informally phrased as: ``every nontrivial semantic
property of partial computable functions is undecidable''.

In the category \(\Asm{\Kone}\) it has the following incarnation:
\begin{exercise}\label{exer:Rice-consequence}
  Show that the exponential \(\Two^\CE \cong \Two^{\pa*{\Sigma^\NatAsm}}\) is isomorphic to \(\Two\).
\end{exercise}

\section{Very first steps in synthetic computability theory}\label{sec:synthetic}

We end this chapter by giving an example of \emph{synthetic computability
theory}~\cite{Bauer2006}.
%
At a high level, the idea is to use the internal logic of \(\Asm{\Kone}\) to
develop computability theory\footnote{Really, the internal logic of the
  realizability topos; see~\cref{chap:topos}.}.
%
The point of restricting to the internal logic is that everything is
automatically computable: you never need to check computability or reason
explicitly about Turing machines for example.
%
A downside (depending on your perspective) is that we have to give up on using
classical logic, because it is not valid in the category as we have seen.

In computability theory, a basic but fundamental result is the following:

\begin{theorem*}[Post]
  If \(A \subseteq \Nat\) and its complement \(\Nat \setminus A\) are computably
  enumerable, then \(A\) is computable.
\end{theorem*}

It is not hard to prove this theorem, but we use it here as an illustration of
what a synthetic development might look like.

\begin{theorem}[Post's theorem, synthetically]\label{Post-synthetically}
  For a realizability predicate \(\phi\) on an assembly \(X \in \Asm{\Kone}\),
  if \(\phi\) and \(\lnot\phi\) are semidecidable, then \(\phi\) is decidable.
\end{theorem}

We prove \cref{Post-synthetically} via two general lemmas.
%
However, the final argument will need one additional logical axiom, namely Markov's
Principle, that is not provable in plain constructive logic, but that it is
valid in the internal logic of \(\Asm{\Kone}\).

\begin{definition}[Markov's Principle]
  The statement that every binary sequence that is not \(0\) everywhere must
  contain a \(1\) is known as \textbf{Markov's Principle}.
  %
  More formally, it is the statement:
  \[
    \forall\pa[\big]{\sigma:\Two^\NatAsm}.  \lnot(\forall(n : \NatAsm).\sigma(n)
    = 0) \Rightarrow (\exists(n : \NatAsm).\sigma(n) = 1).
  \]
\end{definition}

\begin{exercise}\label{exer:Markov's-Principle}\leavevmode
  \begin{enumerate}[(i)]
  \item Show that Markov's Principle is equivalent---over constructive logic---to
    \[
      \forall\pa[\big]{\sigma:\Two^\NatAsm}.  \lnot\lnot(\exists(n :
      \NatAsm).\sigma(n) = 1) \Rightarrow (\exists(n : \NatAsm).\sigma(n) = 1).
    \]
  \item Show that Markov's Principle is realized if and only if every
    semidecidable realizability predicate is \(\lnot\lnot\)-stable.
  \item Show that Markov's Principle is realized over \(\Kone\).
  \end{enumerate}
\end{exercise}

As announced, we now prove two general lemmas. Note that the proofs make no
mention of computability and simply restrict to constructively sound reasoning.

\begin{lemma}\label{semidecidable-closed-under-or}
  If \(\phi\) and \(\psi\) are semidecidable realizability predicates on an
  assembly \(X\), then \(\phi \lor \psi\) is again semidecidable.
\end{lemma}
\begin{proof}
  Let \(x \in \carrier{X}\) be arbitrary.
  %
  We reason purely in constructive logic.
  %
  Suppose there exist binary sequences \(\sigma\) and \(\tau\) such that
  \[
    (\exists(n:\Nat).\sigma(n) = 1) \iff \phi(x)
    \quad\text{and}\quad
    (\exists(n:\Nat).\tau(n) = 1) \iff \psi(x).
  \]
  Then
  \[
    (\exists(n:\Nat).(\sigma\oplus\tau)(n) = 1) \iff \phi(x) \lor \psi(x),
  \]
  where \(\sigma\oplus\tau\) is the binary sequence obtained by taking the
  maximum of the outputs of \(\sigma\) and \(\tau\) at each index.
\end{proof}

\begin{lemma}\label{not-not-decidable}
  For any realizability predicate \(\phi\), we have that
  \(\lnot\lnot(\phi\lor\lnot\phi)\) and \(\top\) are equivalent in the preorder
  of realizability predicates.
\end{lemma}
\begin{proof}
  If \(\phi\) is a realizability predicate on an assembly \(X\), then it
  suffices to show that \(\forall(x : X).\lnot\lnot(\phi(x)\lor\lnot\phi(x))\)
  is realized.
  %
  We show that \(\lnot\lnot(p \lor \lnot p)\) holds generally in constructive
  logic.
  %
  Assume \(\lnot(p \lor \lnot p)\) with the aim of deriving a contradiction.
  %
  Since \(p\) implies \(p \lor \lnot p\), we derive \(\lnot p\).
  %
  But then \(p \lor \lnot p\) holds again which contradicts our assumption.
\end{proof}

We are now ready to prove \cref{Post-synthetically}:
\begin{proof}[Proof of \cref{Post-synthetically}]
  Suppose that \(\phi\) and \(\lnot\phi\) are semidecidable.
  %
  By \cref{semidecidable-closed-under-or} we know that \(\phi\lor\lnot\phi\) is
  again semidecidable.
  %
  Moreover, by Markov's Principle, the predicate \(\phi\lor\lnot\phi\) is
  \(\lnot\lnot\)-stable.
  %
  But \(\forall(x:X).\lnot\lnot(\phi(x)\lor\lnot\phi(x))\) is realized by
  \cref{not-not-decidable}, so we get \(\forall(x:X).\phi(x)\lor\lnot\phi(x)\),
  i.e.\ \(\phi\) is decidable, as desired.
\end{proof}

Once again, we stress the purely logical flavour of the above arguments---with
the exception of checking that Markov's Principle is valid in \(\Asm{\Kone}\)
which should be done once and can then be taken as an additional axiom to the
synthetic development.

Finally, we recover Post's result by specializing \cref{Post-synthetically} to
the assembly of natural numbers and by exploiting the correspondence between the
(semi)decidable realizability predicates and computable/c.e.\ subsets as
explored in
\cref{sec:decidable-realizability-predicates,sec:semidecidable-realizability-predicates}.



\section{List of exercises}
\begin{enumerate}
\item \cref{exer:preorders-monos-predicates-iso}: On monomorphisms as
  realizability predicates and vice versa.
\item \cref{exer:predicates-heyting-prealgebra}: On the Heyting prealgebra of
  realizability predicates.
\item \cref{exer:exists-predicate}: On \(\exists_f\) being left adjoint to
  \(f^\ast\) as maps between preorders of realizability predicates.
\item \cref{exer:epis-monos-logically}: On logical characterizations of
  (regular) epis and monos.
\item \cref{exer:not-double-negation-stable-predicate}: On the fact that not all
  realizability predicates are \(\lnot\lnot\)-stable.
\item \cref{exer:double-negation-stable-iff-regular-mono}: On the
  \(\lnot\lnot\)-stable realizability predicates as the regular monos.
\item \cref{exer:nabla-two-nat-exp}: On exponentials of \(\nabla\set{0,1}\).
\item \cref{exer:nno-in-Asm-K1}: On the natural numbers object in
  \(\Asm{\Kone}\).
\item \cref{exer:decidable-is-computable}: On the correspondence between
  decidable realizability predicates in \(\Asm{\Kone}\) and computable subsets
  of the natural numbers.
\item \cref{exer:Sigma-classifies-semidecidable-realizability-predicates}: On
  the classifier of semidecidable realizability predicates.
\item \cref{exer:Sigma-in-Kleene-1}: On the assembly of semidecidable truth
  values in \(\Asm{\Kone}\).
\item \cref{exer:Sigma-to-N-is-CE}: On computably enumerable subsets and the
  exponential of the assemblies of natural numbers and semidecidable truth
  values in \(\Asm{\Kone}\).
\item \cref{exer:Sigma-ce-subsets}: On computably enumerable subsets and
  pullbacks of the assembly of semidecidable truth values in \(\Asm{\Kone}\).
\item \cref{exer:Rice-consequence}: On Rice's theorem in \(\Asm{\Kone}\).
\item \cref{exer:Markov's-Principle}: On Markov's Principle.
\end{enumerate}

%%% Local Variables:
%%% mode: latexmk
%%% TeX-master: "../main"
%%% End:

\include{mainmatter/topos}

\backmatter%
\printbibliography[heading=bibintoc]%

\end{document}
